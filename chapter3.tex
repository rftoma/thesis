	\chapter {Site Response Analysis}
	\section{Introduction}
	The seismic waves generated at the bedrock suffer variation until they reach the surface; the seismic input as well as the soil material properties influence the non-linear response. The importance of the ground response analysis relates to the overall dynamic performance of the superstructure; the seismic waves can experience amplification or de-amplification at surface level as a function of soil damping properties together with encountered frequencies. It is well known that a system has the tendency to respond at larger amplitude when it vibrates with a frequency matching the natural frequency of the system - the phenomena is known as mechanical resonance and, in structural domain, it translates into high oscillations, thus considerable damage.
	
	In conclusion, performing a site response analysis enables the engineers to numerically predict the site frequency content, the spectral acceleration alongside to peak ground acceleration and thus to create a reliable seismic design. For the current study, the Site Response Analysis is performed in the interest of attaining results that support the next analysis step: the soil-structure interaction.
	
	
	\section{Model description}
	The model represents an extrapolation of the previous analysis step; the soil properties (see Appendix \ref{App:AppendixD}) are applied to a homogeneous layer with properties increasing linearly with depth and it is subjected to two steps: \textit{the gravity} and \textit{the seismic excitation}. Given the lack of bedrock in reality and for the simplicity of calculation, the bedrock is assumed to be located at 30m below surface. In order to simulate a one dimensional analysis, the problem is analysed in a 2D space plane strain ($\epsilon_{zz}=0$) with the lateral width of the soil relatively small compared to the height; the presence of bedrock is numerically equivalent to a rigid base having translational constraints in both directions. Each pair of horizontal nodes are tied together such that will experience equal displacements. Figure \label{fig:Soilcolumn} shows the schematization of the problem, the blue numbers representing the nodes, the red numbers the elements, respectively. The 1D soil column serves as the typical model for a site response analysis as the shear wave propagation direction is assumed to be vertical - thus the main interest relates to the soil performance when experiencing vertically travelling seismic waves.
	
	The verification of the model will consists of results comparison between the analysis performed in Abaqus and the ones in NERA together with the SRA performed by ARUP for Groningen site - the latter only offers a rough guide. The differences between the means of analysis will be investigated, the outcome will be expressed in terms of acceleration response in frequency domain together with stiffness degradation. 
	
	\subsection{Steps of analysis}
	\paragraph{Geostatic} The first step of the analysis simulates the natural stress state within the soil using a gravity load spread all over the layer and it is declared as static, general. The self weight of the elements provide the loads, therefore no need of extra loading conditions. The definition of the gravity load involves the software in updating several parameters to account for the confining pressure. As the schematization shows, the model contains elements and nodes - the nodes will be assigned with boundary conditions reproducing the one dimensional vertical propagation of the shear waves. The base nodes are constrained in both directions in accordance with the assumption of the bedrock presence underneath the soil layer whilst each pair of nodes horizontally displayed are linked using a feature called multi-point constraint - this provides a pinned joint between the nodes making the global displacements be equal while leaving the rotations independent. 
	\begin{figure}
		\centering
		\includegraphics[width=0.6\linewidth]{"Soil column"}
		\caption[]{Schematization of one-dimensional soil column concept (right) and representation of the SRA model in Abaqus highlighting the position of nodes}
		\label{Soilcolumn}
	\end{figure}
	
	\paragraph{Dynamic}
	The second step represents the soil medium subjected to the earthquake motion; it is defined as dynamic step using an implicit integration scheme which will be explained in more detail later on. During the step, the user propagates the boundary conditions aforementioned with the exception of the horizontal base constraint; it is replaced by an acceleration type of boundary condition that is assigned with an amplitude curve - this allows the user to define an arbitrary variation in time (or frequency) of any quantity or can be defined as a mathematical function (e.g. sinusoidal), series of values in time (acceleration history) and many more. For the current study, the amplitude curve contained the acceleration-time record for a period of 10 seconds as \ref{Acceleration} shows; this period is considered as the representative duration of Groningen acceleration measured time history and it starts right before the peak acceleration \cite{dost2004scaling}. 
	
	\begin{figure}[h!]
		\centering
		\includegraphics[width=0.8\linewidth]{"Acc input"}
		\caption{Acceleration input (induced earthquake, WSE station, Huizige, Groningen, 16 Aug 2012). Acceleration recorded at borehole depth}
		\label{Acceleration}
	\end{figure}
	
	\newpage
	\subsection{Input parameters}
	The material parameters can be divided into several categories the software module requires:
	\begin{enumerate}
		\item \textit{Elastic}\quad again, Young's modulus together with Poisson's ratio becomes sufficient for the definition of elastic domain.
		\item \textit{Plastic}\quad the nonlinear isotropic/kinematic model applies von Mises failure criterion for which three parameters are required: yield stress at zero plastic strain $\sigma_0$, small-strain Young's modulus $C$ and hardening parameter $K$.
		\item\textit{Density -}\quad the soil layer is considered to be homogeneous, thus one single material is define. So density for a silty-clay medium weak was considered 1800kg/m3
		\item \textit{Damping -}\quad the software offers many damping options [from numerical to material damping]; however, for this example the Rayleigh damping parameters $\alpha$ and $\beta$ are sufficient for the scope of analysis. This matter will be discussed later on.
	\end{enumerate}
	
	In addition, the model contains a subroutine which introduces the values of undrained shear strength as a function of the vertical effective stress as well as the correlation of this parameter with Young's modulus. The benefits of using a subroutine relate the expansion of the standard input parameter set by inserting field variables at material points as functions of time or of any material quantity. Moreover, it links the field variables with solution-dependent material properties, values being updated at the start of each increment through interpolation; an average value is used for linear elements; an approximate linear variation is used for quadratic elements. The routine allows the user to access quantities at material points and manipulate them according to the desired application. 
	
	Appendix \ref{App:AppendixG} shows the flow chart of Abaqus user defined subroutine whilst Appendix \ref{subroutine} shows the routine created for these calculations. The input parameters are calculated as presented in the previous chapter (see Chapter \ref{ch3}, section 3.4 and Appendix \ref{App:AppendixF}).  
	
	\subsection{Rayleigh coefficients}
	The dissipative character of the elasto-plastic material can be better enhanced by introducing a viscous damping mechanism. The overall stress state of the soil skeleton can be separated into two parts: 
	\begin{itemize}
		\item \textit{frictional component}- relating to the elasto-plastic behaviour, being displacement proportional
		\item \textit{viscous component} - relating to the Rayleigh damping coefficients, being velocity proportional. 
	\end{itemize}
	
	The advantage of coupling these two terms refers to the smoothing of the shear stress-strain cycles as well as a higher material damping. The viscous component has a favourable effect on the damping curves for small-strain domain; whilst it leads to an overestimation for large-strains. 
	As mentioned previously, the second step of the analysis, and the most important, is declared as dynamic, implicit. In non-linear analysis, the dynamic equation of motion is:
	\begin{equation}
	\left[M\right]{\Delta \ddot{u}}+\left[C\right]{\Delta \dot{u}}+\left[K\right]\Delta{u}=-\left[M\right]{I}\Delta\ddot{u}_g
	\end{equation}
	
	where \gls{M} - mass matrix, \gls{C} - viscous damping matrix, \gls{Kk} - stiffness matrix, $\Delta \ddot{u}$, $\Delta \dot{u}$ and $\Delta{u}$ - vectors of nodal relative acceleration, velocity and displacement, $\Delta \ddot{u}_g$ - acceleration at the base and {I} - unit vector. The matrices are compiled based on the incremental response recorded within the software analysis. The equation is solved at every incremental time step using the Newmark \cite{newmark1959method} method.
	
	The viscous damping matrix follows Rayleigh and Lindsay, 1976 \cite{rayleigh1976theory} equation that relates the small strain damping to both mass and stiffness matrices. Thus, the equation for the viscous damping is:
	\begin{equation}
	\left[C\right]=\alpha\left[M\right]+\beta\left[K\right]
	\end{equation}
	where 
	\begin{equation}
		\xi = \frac{1}{2}(\beta \omega + \frac{\alpha}{\omega})
	\end{equation}
	\gls{alpha} and \gls{betaa} are determined by choosing fractions of the critical damping $\xi_{crit}$. Assuming two different circular frequencies $\omega_1$ and $\omega_2$ with the corresponding damping ratios $\xi_1$ and $\xi_2$, the two parameters become:
	\begin{equation}
		\alpha=\frac{2 \omega_1 \omega_2 (\xi_1 \omega_2 - \xi_2 \omega_1)}{\omega_2^2 - \omega_1^2}
	\end{equation}
	\begin{equation}
		\beta=\frac{2 (\xi_1 \omega_2 - \xi_2 \omega_1)}{\pi (\omega_2^2 - \omega_1^2)}
	\end{equation}
	
	Small strain viscous damping effects are assumed proportional only to the stiffness of the soil layers because the soil medium is regarded as relatively short, thus the contribution of higher modes becomes negligible. Additionally, for cases with low frequencies, the $\omega_1$ is the related to the first natural mode while the second relevant mode can be considered zero (and $\omega_2 = 0$). Thus, $\alpha$ parameter is set to zero, whilst $\beta$ yields:
	
	\begin{equation}
	\beta = 2\frac{\xi}{\omega_1}
	\end{equation}
	
	
	$\xi$ is the damping ratio at small strain; for this application $\xi$=1.2 corresponding to Darendeli's results; while \gls{omega} is the natural circular frequency at first natural mode;
	
	Despite that the experimental results prove that the damping matrix [C] is frequency independent and it has a constant value, Park and Hashash, 2002 \cite{hashash2002viscous} show the correlation between the stiffness [K] and the damping [C]. Since the stiffness depends on the strain level, then it can be concluded that the damping is also strain dependent, including the natural frequency. Thus, the viscous damping matrix [C] is updated at each time increment as well as the stiffness. Kramer, 1996 \cite{kramer1996geotechnical} proposes an equation for the calculation of the period of vibration that corresponds to the fundamental frequency of the selected mode as it follows:
	\begin{equation}
	T_n=(2n-1)\frac{\stackrel{-}{{V}_{s}}}{4H}\longrightarrow T_n=\frac{V_s}{4H}.
	\end{equation}
	
	where $\stackrel{-}{{V}_{s}}$ average shear wave velocity and H = layer height.
	
	Further, the natural circular frequency is obtained through the following transformation:
	\begin{equation}
	\omega_n=\frac{2\pi}{T_n} \rightarrow \omega_1=\pi\frac{V_s}{2H}
	\end{equation}
	
	And finally, Rayleigh $\beta$ damping parameter becomes:
	\begin{equation}
	\beta=\frac{2H\xi}{V_s\pi}
	\end{equation}
	
	\begin{table}[h!]
		\centering
		\begin{tabular}{|c|c|c|c|c|}
			\hline $\xi$         &       $1.2$    &  \%    &  Darendeli damping ratio      \\ 
			\hline $T_n$    & $4H/V_s$ &  sec &  Fundamental period\\ 
			\hline $V_s$  & 132  &  m/s &  Average shear wave velocity\\ 
			\hline $H$ & 30 &  meters &  Layer depth\\
			\hline $\omega$ & 6.91 & $Hz (sec^-1)$ & Fundamental frequency\\
			\hline $\beta$ & 0.001736 & sec & Rayleigh $\beta$ coefficient\\
			\hline
		\end{tabular} 
		\caption{Calculation of Rayleigh $\beta$ parameter as a function of the damping ratio and eigen frequency}
		\label{beta_param}
	\end{table}
	
	\subsection{Integration scheme} \label{sectionNew}
	It is important to understand the type of nonlinear analysis the software was opted to perform and its influence of the results. For this particular analysis, an implicit direct integration scheme was selected - this means that the set of nonlinear equations of motion solved at the time step $\Delta t_{n+1}$ are employed to compute the transition from the state at $t_n$ to $t_{n+1}$; on the other hand, an explicit integration uses all information at the beginning of $t_n$ to estimate the latter state at $t_{n+1}$.
	
	Abaqus/Standard uses the Hilber-Hughes-Taylor time integration method \cite{hilber1977improved} which is an extension of the Newmark $\beta$ method, 1959 \cite{newmark1959method}. Basically it controls the numerical damping within the system which might rise due to the energy dissipation mechanisms associated with different operator types. More details related to Newmark integration method are presented in Appendix \ref{App:AppendixH}.
	
	\subsection{FEM stability}
	According to Jeremi\'{c}, \cite{jeremic2009time}, the accuracy of such nonlinear problem dealing with wave propagation is controlled by two factors - node spacing in the FE model $\Delta h$ and the time step $\delta t$. The spacing is directly related to the wavelength whilst the time step depends on the fundamental period of the system. 
	\begin{equation}
	\Delta h\leq\frac{\lambda_{min}}{10}=\frac{1}{10} \frac{v}{f_{max}}
	\end{equation}
	
	\begin{equation}
	\Delta t=\frac{T_n}{10}
	\end{equation}
	More details regarding the calculation of the two stability criterion are presented within Appendix \ref{App:AppendixI}
	
	\subsection{Mesh}
	For the meshing part, structured four-node continuum elements ($CPE4$) are used to model the soil using the plane strain formulation of the quad-element. The element connectivity uses a counterclockwise pattern for the previously-described node numbering scheme (see \ref{Figure 1}). The soil elements in each layer are assigned the material tag corresponding to that layer. A unit thickness is used in all examples for simulating the 1D condition. The self-weight of the soil is considered as a body force acting on each element. The body force is set as the unit weight of the soil in each layer, which is determined from the respective mass density input value.
	
	
	\section{NERA - Non-linear Earthquake Site Response}
	An additional analysis was performed using NERA (Nonlinear Earthquake site Response Analyses) \cite{NERA} computer program in order to have a better grasp of the whole concept as well as a meaningful comparison. It derives from EERA that uses an equivalent linear model to investigate the same problem; the constitutive material model is proposed by Iwan and Mroz, 1967 \cite{mroz1967description} and it describes the nonlinear kinematic model as a series of n mechanical elements, each displaying different stiffness ki and sliding resistance $R_i$ as it is shown in Figure \ref{Mroz}. More details concerning the calibration NERA performs are presented in Appendix \ref{NERAa}. %scrie NERA.
	
	\begin{figure}[h!]
		\centering
		\includegraphics[width=0.7\linewidth]{"Mroz"}
		\caption[]{Plasticity description of Iwan and Mroz nonlinear model used as constitutive model in NERA software.}
		\label{Mroz}
	\end{figure}
	
	NERA is composed by a FORTRAN 90 code together with a Excel plug-in that capture the nonlinear site response of a layered soil column subjected to an earthquake signal. As input, it requires the acceleration-time record, elastic material properties with emphasis on the stiffness parameters - normalized stiffness degradation and material damping curves. 
	
	The material properties introduced within the software require a general description of the soil medium as well as an individual material classification (in case the user is dealing with a multi-layered ground). The general input portrays the column with its sub-layers and unit weights, shear wave velocities and the ground water table followed by an automatic assessment of the maximum shear modulus, effective stresses and fundamental period. One example of input parameters can be seen in Appendix \ref{Input_NERA}.
	
	The software allows for multiple layer definition together with the normalized shear modulus degradation curves and material damping curves. One example of output can be seen in Appendix... For this particular case, Darendeli's results presented previously were used in order to obtain a consequent comparison. In the end, the two numerical analyses (ABAQUS and NERA) aim towards the same goal; thus, the standard normalized set of curves proposed by Darendeli serves as benchmark for non-linear behaviour. However, the software does not account for data corresponding to damping curves as it is calculating its own set of values. This is because the nonlinear model proposed by Iwan and Mroz uses a different scheme for determining the damping characteristics.
	
	NERA starts the analysis by processing the earthquake input signal and material properties followed by step-by-step determination of the relative velocity and displacement at a selected sub-layer. The integration scheme NERA uses is the central difference which, contrary to Abaqus analysis, it is conditionally stable (the time step size is limited). This calculation method is a particular type of Newmark method in which the predicted velocity is:
	
	\begin{equation}
	\tilde{v}_{i,n+1}=v_{i,n}+\frac{1}{2}a_{i,n}\Delta t
	\end{equation}
	
	The stress and strains are calculated at each node from nodal displacements, next step calculates the velocity from the input acceleration history. Subsequently, the predicted values for velocities at time $t_{n+1}$ yield from those determined at time $t_n$ and finally the nodal displacements, velocities and accelerations are update at each node i. In addition, the output includes spectral response based on Fourier transform calculation for any selected sub-layer.
	
	\section{Results}
	This subchapter presents the results obtained from the site response analysis performed in Abaqus, followed by a comparison with NERA outcome; several plots are used as mean of verification and the main interest lies on the $G/G_{max}$ and damping curves, acceleration response as well as the amplification ratio in frequency domain.
	
	After performing a dynamic, implicit type of analysis to the assembly described previously, the output was investigated with an initial emphasis on the stiffness degradation. To check the validity of the overall model, an uniform cyclic signal served as acceleration amplitude; a circular frequency $f=1 Hz$ was applied at the bottom of the model for a period of 10 seconds in a sinusoidal fashion. Such uniform harmonic loading represents a simplification of the signal, and helps the user better understand the non-linear soil response. A real acceleration-time history consists in many irregularities and the results may be sometimes misinterpreted. That is why the investigation starts simple towards complicated.
	
	The results can be visualized in figures \ref{response1}, \ref{response2} and \ref{acc1}. The same remarks can be made as for the preceding analysis (see previous analysis in Chapter \ref{ch3}); however, this model includes an additional Rayleigh damping parameter together with the subroutine that updates the values of the Young's modulus as a function of the undrained shear strength and effective stresses, respectively. The acceleration response displays a de-amplification of the bottom signal together with damping effects visible in the irregularities of the curves; additionally, the average wave speed decreases (blue line wave travels with 0.5m/s whilst the green line with almost half the speed). The values are obtained by dividing the maximum acceleration to the time period of a complete sinusoidal cycle. The shear stress-strain plots show plasticity occurring at different stress levels - the hysteresis loops shift towards origin which means that the soil achieves plasticity gradually together with residual plasticity. Additionally, the plots show a decrease in dissipated energy as the seismic waves travels from bottom to top - the area comprised within the hysteresis loops reduces as well, which comes in conformity with the conclusions of the previous chapter (see Chapter \ref{ch3})
	
	%introduce the soil column picture
	\begin{figure}[h!]
		\centering
		\includegraphics[width=0.7\linewidth]{"response1"}
		\caption[]{Hysteresis loops for different elements situated across the lower half of the soil layer (Element 1 - Base; Element 15 - Middle of the layer)}
		\label{response1}
	\end{figure}
	
	\begin{figure}[h!]
		\centering
		\includegraphics[width=0.7\linewidth]{"response2"}
		\caption[]{Hysteresis loops for different elements found in the upper half of the soil layer (Element 1 - Base; Element 31 - Surface)}
		\label{response2}
	\end{figure}
	
	\begin{figure}[h!]
		\centering
		\includegraphics[width=0.6\linewidth]{"acc_response1"}
		\caption{Acceleration response for the cyclic input signal (surface and bottom). The base input corresponds to an uniform harmonic signal of frequency f=1Hz lasting for t=10 seconds}
		\label{acc1}
	\end{figure}
	
	Once the validation of this simplistic model is achieved, the real acceleration-time history is applied to the boundary condition at the bottom nodes for the same period of 10 seconds. The records correspond to the measurements from WSE station, Huizinge, Groningen \cite{dost2013august}. Even though the earthquake signal lasts for 35 seconds, it was trimmed to incorporate the most unfavourable 10 seconds [with the highest acceleration peak] which, according to ARUP \cite{martellotta2015review}, are appointed as the representative duration of Groningen acceleration measured time history starting just before the peak acceleration. The obtained values in terms of shear stress and strain can be seen in Figure \ref{resp3} whilst the top and bottom acceleration outcome is represented in Figure \ref{acc_resp2}. Important to mention is that the boundary condition applied at the bottom of the model simulates the bedrock presence, the waves being incident only - the discussion about the boundary conditions influence on the wave propagation shall be presented shortly after. 
	
	\textit{Note: As a sign convention, element 1 represents the bottom one whilst element 30 is the surface.}
	\begin{figure} [h!]
		\centering
		\includegraphics[width=0.7\linewidth]{"response3"}
		\caption{Hysteresis loops for upper half of the layer when real acceleration is applied. The accelerogram corresponds to the real one recorded in Huizinge and the response was extracted at various depths}
		\label{resp3}
	\end{figure}
	
	\begin{figure}[h!]
		\centering
		\includegraphics[width=0.7\linewidth]{"acc_response2"}
		\caption{Acceleration response - surface and bottom}
		\label{acc_resp2}
	\end{figure}
	%introdu citat
	The response obtained at base and surface level is then converted into frequency domain via the Fast Fourier Transform (FFT). As Kristeková et al., 2006 \cite{kristekova2006misfit} stated "the most complete and informative characterization of a signal can be obtained by its decomposition in the time-frequency plane". However, FFT is used as a digital signal processing tool that aids other operation, rather than providing a final result itself. For example, the transfer function or a site amplification factor allows the calculation of the motion of any layer $i$ based on other motion (of any layer $j$). This transfer function relates the displacement at position $i$ to that at position $j$ as it follows:
	\begin{equation}
	F_{ij}=\frac{|u_i|}{|u_j|}
	\end{equation}
	
	The Fourier spectrum represents the distribution of energy in the ground motion for a frequency interval $[0 ≤ f ≤ 1/2\Delta t]$. The amplification factor can also be written as:
	\begin{equation}
	A(f)=\frac{F_{a,site}(f)}{F_{a,bedrock}(f)}
	\end{equation}
	where $A(f)$ - site amplification factor; $F_a,site(f)$ and $F_a,berock(f)$ - Fourier amplitudes of ground acceleration at surface and bedrock, respectively calculated as the vector sum of the two horizontal components. The importance of the site amplification factor relates to the structural damage pattern (rather than the PGA amplification) due to site effects. If the natural period of the structure matches the soil natural period, the structure will be more vulnerable to earthquakes. 
	\begin{figure}[h!]
		\centering
		\includegraphics[width=0.7\linewidth]{"Fourier"}
		\caption{Fourier amplitude spectra corresponding to ABAQUS numerical analysis recorded at surface and bottom level.}
		\label{fourier}
	\end{figure}
	%alta referinta Zienkiewicz
	
	It is important to discuss the boundary conditions and their influence on the wave transmission throughout the layer. The current model created in Abaqus includes tied boundaries, also described by Zienkiewicz et al. (1989) \cite{zienkiewicz1989earthquake} and it assumes that all displacements on the left side of the column correspond to the ones at the right side, simulating a free field boundaries, whereas the bottom nodes do not absorb the oscillating waves. The software offers the baseline correction option for the acceleration input for time domain analysis; the practice is proposed by Newmark \cite{newmark1959method} and it introduces an additional correction to the acceleration such that the mean square velocity over the event time is minimized. The use of more correction intervals provides tighter control over any “drift” in the displacement at the expense of more modification of the given acceleration trace. However, in acceleration history it does not show large difference between the responses.
	
	Because the boundary will transmit the waves only upwards, an additional analysis was performed with a bottom acceleration record manually reduced by half. Physically, the waves within a soil have both downwards and upwards motion (incident and reflective). The decision of applying half of the signal also relates to NERA procedure as it assumes that the velocity at base is the sum of both incident and reflected waves and since at bedrock level shear force is zero it yields that the velocity becomes:
	
	\begin{equation}
	v_{base}=2v_{incident}
	\end{equation} 
	
	\newpage
	\section{Comparison Abaqus vs NERA}
	The comparison refers to the inelastic response focusing on the stiffness, acceleration and spectral response. Moreover, the amplification factor in investigated. The outcome of the two different PGA levels are also presented and the plots can be visualized below. 
	
	\begin{figure}[h!]
		\centering
		\includegraphics[width=0.7\linewidth]{"acc_comp1"}
		\caption{Acceleration response at surface level - comparison Abaqus vs. NERA. PGA=0.6g lasting for a period of t=10 seconds}
		\label{comp1}
	\end{figure}
	
	\begin{figure}[h!]
		\centering
		\includegraphics[width=0.6\linewidth]{"spectral2"}
		\caption{Fourier amplitude spectra corresponding to results extracted from ABAQUS (green) and NERA (black) at surface and base levels for PGA level of 0.6g. In general, Abaqus produces higher (plastic) strains resulting in more attenuated peaks in the frequency domain compared to NERA}
		\label{fourier2}
	\end{figure}
	
	\begin{figure}[h!]
		\centering
		\includegraphics[width=0.6\linewidth]{"tau_gamma1"}
		\caption{Hysteresis loops comparison between NERA (blue) and ABAQUS(red) for the response at surface level. The results show a correspondence in the computed shear modulus, both tangent ($G_{tan}$) and maximum/small strain ($G_{max}$) between the two software}
		\label{tau_gamma1}
	\end{figure}
	
	\begin{figure}[h!]
		\centering
		\includegraphics[width=0.85\linewidth]{"tau_gamma2"}
		\caption{Hysteresis loops comparison between the software for various sample depths: z=20m(left) and bottom level(right). Both Abaqus and NERA compute the shear modulus in similar manner (see dashed lines)}
		\label{tau_gamma2}
	\end{figure}
	
	\newpage
	\subsection{Additional investigation}
	An additional investigation was performed based on a lower PGA value (PGA=0.1g). This first reason relates to the soil response when closer to elastic domain. Secondly, to highlight a difference in dynamic behaviour for various levels of base acceleration. Moreover, it is speculated that lower PGA characteristics might act detrimental compared to a higher value - a rather seismic wave amplification is expected from such input. Lastly, the recorded accelerograms in Groningen showed an overall low PGA value. The results can be seen in the following figures.
	
	\begin{figure}[!h]
		\centering
		\includegraphics[width=0.7\linewidth]{"acc_low"}
		\caption{Acceleration response comparison (NERA vs ABAQUS) for a PGA=0.1g. The plot incorporates the cyclic signal injected at the base of the model as well as the surface response extracted from both the software.}
		\label{acc_low}
	\end{figure}
	
	\begin{figure}[!h]
		\centering
		\includegraphics[width=0.6\linewidth]{"hysteresis_low_bot"}
		\caption{Hysteresis loops comparison for bottom level for a PGA level of 0.1g. The dashed lines highlight the computed shear modulus which is practically the same for both the software.}
		\label{hyst_bot}
	\end{figure}
	
%	\begin{figure}[!h]
	%	\centering
	%	\includegraphics[width=0.6\linewidth]{"hysteresis_low_surf"}
	%	\caption{Hysteresis loops comparison for surface level for PGA=0.1g. It is rather difficult to obtain a meaningful comparison at surface due to the non-linear nature of the problem.}
	%	\label{hyst_top}
%	\end{figure}
	
	\begin{figure}[!h]
		\centering
		\includegraphics[width=0.6\linewidth]{"spectral_low"}
		\caption{Fourier amplitude spectra for PGA=0.1. The results are ploted on the logarithmic scale as it is common practice in the literature. The results were obtained by converting the response from both base and surface levels from time into frequency domain using FFT.}
		\label{fourier3}
	\end{figure}
	
	\begin{figure}[!h]
		\centering
		\includegraphics[width=0.6\linewidth]{"site_ampl"}
		\caption{Site amplification factor comparison for PGA=0.1g (black line - NERA, green - ABAQUS).}
		\label{SAF}
	\end{figure}
	
	\begin{figure}[!h]
		\centering
		\includegraphics[width=0.7\linewidth]{"ARUP"}
		\caption{Groningen site non-linear ground response, input vs surface PGA according to ARUP report; the additional blue star corresponds to the present high PGA analysis (PGA=0.6g) and the yellow star to the low PGA case (PGA=0.1g). The red line marks the borded between the two regions - amplification and de-amplification.}
		\label{ARUP}
	\end{figure}
	
	The results confirm the expectation regarding the amplification effect of the soil combined with the PGA level, also they are in accordance with ARUP report regarding the site response analysis performed for the specific location in discussion (see figure \ref{ARUP}). The acceleration responses at surface produced in both the computational software present the same maximum value, of course, with differences that were exposed previously, the elastic response seems to match together with the stiffness degradation effect. 
	\pagebreak
	
	
	\newpage
	\subsection{Remarks}
	It is difficult to have a relevant comparison at surface level when the sample experiences low stresses; calibration of cyclic parameters was referred to rather high confining stresses. Moreover, it was noticed that Abaqus generates higher plastic strains; this might explain the signal attenuation expressed in frequency domain when comparing to NERA response. 
	
	Overall, the results show a quite satisfying match; nonetheless, differences appear due to several factors as it follows:
	\begin{enumerate}
		\item Abaqus works with a dynamic implicit integration scheme with an Euler backwards theory that is unconditionally stable whereas NERA solves the system of equation through a central-difference integration which is conditionally stable; thus on one software, user has the option of manually adjusting the time step while on the other he cannot. The equation of motion for the system in dynamic domain it is basically the same:
		\begin{equation}
		f=F-\rho\ddot{u}
		\end{equation}
		where f - body force, F - external body force, $\rho$ - soil unit weight and $\ddot{u}$ - relative nodal acceleration. Thus, the difference is \textit{numerical} and cannot be adjusted in detail.
		\item Both the models are based on the concept of a yield surface which means that there is a surface defined in stress space within which no plastic deformation takes place. The nonlinear kinematic hardening model used in Abaqus is based on the Lemaitre and Chaboche \cite{lemaitre1994mechanics} model that considers two yield surfaces and a varying hardening modulus whilst NERA works according to Iwan and Mroz \cite{mroz1967description} multi-linear kinematic hardening model that has a multiple surface plasticity. Both models include a 'fading memory' term but function differently. Thus, the difference is related to the plastic domain and how the model behaves when irreversible deformations are encountered. 
		\begin{figure}[!h]
			\centering
			\includegraphics[width=0.7\linewidth]{"yield_srf"}
			\caption{Non-linear kinematic hardening model according to Lemaitre,Chaboche [left] and Iwan,Mroz[right]}
			\label{Yield}
		\end{figure}
		\item NERA computes the shear stress and strain increments according to the sliding resistance and tangent modulus of each slider that are derived from the $G_i-\gamma_i$ points introduced as input. The tangential shear modulus is related to the secant shear modulus by $H_i=G_{max}\frac{G'_{i+1}\gamma_{i+1}-G'_i\gamma_i}{\gamma_{i+1}-\gamma_i}$ where $G_i=\frac{G_i}{G_{max}}$ . The stiffness of each component of the system is related to the tangent modulus while the shear stress is associated with the sliding resistance. On the other hand, Abaqus does not work with the tangent modulus, but with maximum shear modulus Gmax and the undrained shear strength. Also, in NERA there is no option for introducing the values corresponding to the undrained shear strength or its correlation with any other parameter. Figure 14 shows the distribution of the shear modulus with depth in both the software - these values are calculated from the input parameter - Abaqus extracts it based on the Young's modulus together with Poisson's ratio whilst NERA does it based on the soil unit weight and shear wave velocity. The values are exactly the same, it can be noticed also from the slope the hysteresis loops presented in the stress-strain plots (see fig. \ref{shear1}); the width also resembles as well as the unloading-reloading areas. However, the tangent shear modulus differs from one software to the other.

		\item Abaqus results are extracted in two different locations: the stresses and strains are obtained from Gauss integration points and the accelerations are obtained at nodal positions which are situated at the interface of each two element. NERA performs the analysis according to the middle of each sub-layer, so both the top and bottom outcome are actually 0.25m below the surface or above the bottom. However, not much difference should rise from this aspect.
		\item Another difference relates to the viscous part of the model and it is more apparent in the acceleration response in time domain, especially in the initial period. Abaqus works with the viscous term defined exclusively as a function of Rayleigh damping parameter $\beta$ meaning that the system damping depends on the stiffness, whilst NERA requires the critical damping ratio value, incorporating both $\alpha$ and $\beta$ Rayleigh factors. Moreover, the energy dissipation mechanism accounts for both viscous damping and plasticity terms which, as described previously, differ from NERA to Abaqus. When working with a high PGA value, the contribution of the non-linear model and hysteretic damping becomes greater compared to the viscous damping component.
		\item All these mentioned points that highlight the difference between the two software derive mainly from numerical algorithms; however, a more important aspect seems to govern the outcome of each analysis - the dynamic nature of the problem itself. The two models are not identical, thus a difference in any given point might generate distinct response at unknown location and time because of the non-linear propagation fashion. 
		\item Noise can be experienced in both the signals of the different computer programs. This can distort the final results; filtering is possible, however it is time demanding and does not represent the main interest in this study.
	\end{enumerate}
			\begin{figure}[h!]
				\centering
				\includegraphics[width=0.5\linewidth]{"shear"}
				\caption{Shear modulus $G_{max}$ comparison between the input values used for the numerical analysis performed in ABAQUS (red) and the computed values within NERA(blue). The two software work with exactly the same values.}
				\label{shear1}
			\end{figure}
	
	\newpage
	\section{Conclusions}
	This chapter assemblies the soft soil conditions and a recorded acceleration time history from the specific site. A homogeneous clay layer of 30m height was subjected to a dynamic load. The seismic input corresponds to the a real accelerogram recorded in Groningen, in August 2012. The soil characteristics were empirically calculated and they correspond to the values presented in the previous chapter. Additionally, the Rayleigh damping parameters were incorporated in the material module together with a subroutine which updates the undrained shear strength and Young's modulus as a function of the vertical effective stresses. One-dimensional domain was considered due to the assumption of vertically propagated shear waves. A dynamic implicit integration scheme was used for processing the dynamic response: a special Newmark $\beta$ method (Hughes-Hilber-Taylor) describes the integration process. The stability of the analysis was also addressed. For validation purpose, a few additional site response analyses were performed with NERA - a software specially designated for such dynamic applications. Two values of PGA (peak ground amplitude) were considered depicting two different scenarios: strong and medium earthquake signals. The outcome focused on stiffness degradation, damping, acceleration response at surface levels and wave amplification.
	
	Thus, the current study applied various numerical methods to achieve same output and proceeds with a comparison between the analyses. It acknowledged, in the same time, the differences and limitations occurring amid. Given the fact that the study deals with a non-linear kinematic hardening process, it becomes quite difficult to point exactly the main cause that leads to dissimilarity. The overall results showed a fair match when comparing the two software, but also when accounting for external sources (KNMI, \cite{dost2012monitoring}, \cite{dost2004scaling}, \cite{dost2013august}) which carried out particular investigations. 
	
	It can be concluded that for the Groningen sites, for a low PGA input ($a_g=0.1g$) the surface acceleration response amplified whereas when dealing with a high PGA input ($a_g\textgreater0.15g$) the top PGA is de-amplified. Both the strength properties of the homogeneous medium and the earthquake motion characteristics influence the seismic wave propagation through the soil; for this particular case, the inelastic response depends on the limited capability of the soft layers to transmit seismic signals to the surface because of the low strength and on the high hysteretic damping observed at large strains. At low PGA values, the soil displays elastic behaviour as it reaches the surface, the frequencies do not experience considerable damping whereas at high PGA levels, the soil shifts towards its natural frequency, damping the higher ones and non-linear constitutive model plays a more important role than the viscous damping effect.