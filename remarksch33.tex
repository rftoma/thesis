\subsection{Remarks}
	It is difficult to have a relevant comparison at surface level when the sample experiences low stresses; calibration of cyclic parameters was referred to rather high confining stresses. Moreover, it was noticed that Abaqus generates higher plastic strains; this might explain the signal attenuation expressed in frequency domain when comparing to NERA response. 
	
	Overall, the results show a quite satisfying match; nonetheless, differences appear due to several factors as it follows:
	\begin{enumerate}
		\item Abaqus works with a dynamic implicit integration scheme with an Euler backwards theory that is unconditionally stable whereas NERA solves the system of equation through a central-difference integration which is conditionally stable; thus on one software, user has the option of manually adjusting the time step while on the other he cannot. The equation of motion for the system in dynamic domain it is basically the same:
		\begin{equation}
		f=F-\rho\ddot{u}
		\end{equation}
		where f - body force, F - external body force, $\rho$ - soil unit weight and $\ddot{u}$ - relative nodal acceleration. Thus, the difference is \textit{numerical} and cannot be adjusted in detail.
		\item Both the models are based on the concept of a yield surface which means that there is a surface defined in stress space within which no plastic deformation takes place. The nonlinear kinematic hardening model used in Abaqus is based on the Lemaitre and Chaboche \cite{lemaitre1994mechanics} model that considers two yield surfaces and a varying hardening modulus whilst NERA works according to Iwan and Mroz \cite{mroz1967description} multi-linear kinematic hardening model that has a multiple surface plasticity. Both models include a 'fading memory' term but function differently. Thus, the difference is related to the plastic domain and how the model behaves when irreversible deformations are encountered. 
		\begin{figure}
			\centering
			\includegraphics[width=0.7\linewidth]{"yield_srf"}
			\caption{Non-linear kinematic hardening model according to Lemaitre,Chaboche [left] and Iwan,Mroz[right]}
			\label{Yield}
		\end{figure}
		\item NERA computes the shear stress and strain increments according to the sliding resistance and tangent modulus of each slider that are derived from the $G_i-\gamma_i$ points introduced as input. The tangential shear modulus is related to the secant shear modulus by $H_i=G_{max}\frac{G'_{i+1}\gamma_{i+1}-G'_i\gamma_i}{\gamma_{i+1}-\gamma_i}$ where $G_i=\frac{G_i}{G_{max}}$ . The stiffness of each component of the system is related to the tangent modulus while the shear stress is associated with the sliding resistance. On the other hand, Abaqus does not work with the tangent modulus, but with maximum shear modulus Gmax and the undrained shear strength. Also, in NERA there is no option for introducing the values corresponding to the undrained shear strength or its correlation with any other parameter. Figure 14 shows the distribution of the shear modulus with depth in both the software - these values are calculated from the input parameter - Abaqus extracts it based on the Young's modulus together with Poisson's ratio whilst NERA does it based on the soil unit weight and shear wave velocity. The values are exactly the same, it can be noticed also from the slope the hysteresis loops presented in the stress-strain plots; the width also resembles as well as the unloading-reloading areas. However, the tangent shear modulus differs from one software to the other.
		\begin{figure}[h!]
			\centering
			\includegraphics[width=0.7\linewidth]{"shear"}
			\caption{Shear modulus $G_{max}$ comparison}
			\label{shear1}
		\end{figure}
		\item Abaqus results are extracted in two different locations: the stresses and strains are obtained from Gauss integration points and the accelerations are obtained at nodal positions which are situated at the interface of each two element. NERA performs the analysis according to the middle of each sub-layer, so both the top and bottom outcome are actually 0.25m below the surface or above the bottom. However, not much difference should rise from this aspect.
		\item Another difference relates to the viscous part of the model and it is more apparent in the acceleration response in time domain, especially in the initial period. Abaqus works with the viscous term defined exclusively as a function of Rayleigh damping parameter $\beta$ meaning that the system damping depends on the stiffness, whilst NERA requires the critical damping ratio value, incorporating both $\alpha$ and $\beta$ Rayleigh factors. Moreover, the energy dissipation mechanism accounts for both viscous damping and plasticity terms which, as described previously, differ from NERA to Abaqus. When working with a high PGA value, the contribution of the non-linear model and hysteretic damping becomes greater compared to the viscous damping component.
		\item All these mentioned points that highlight the difference between the two software derive mainly from numerical algorithms; however, a more important aspect seems to govern the outcome of each analysis - the dynamic nature of the problem itself. The two models are not identical, thus a difference in any given point might generate distinct response at unknown location and time because of the non-linear propagation fashion. 
		\item Noise can be experienced in both the signals of the different computer programs. This can distort the final results; filtering is possible, however it is time demanding and does not represent the main interest in this study.
	\end{enumerate}
	
	
	\section{Conclusions}
	This chapter assemblies the soft soil conditions and a recorded acceleration time history from the specific site. A homogeneous clay layer of 30m height was subjected to a dynamic load. The seismic input corresponds to the a real accelerogram recorded in Groningen, in August 2012. The soil characteristics correspond to the values calculated in the previous chapter and are determined empirically. Additionally, the Rayleigh damping parameters are incorporated in the material module together with a subroutine which updates the undrained shear strength and Young's modulus as a function of the vertical effective stresses. The whole system is considered acting as one-dimensional since the shear waves propagate only vertically. A dynamic implicit integration scheme was used for processing the dynamic response: a special Newmark $\beta$ method describes the integration process. The stability of the analysis was also addressed. For validation purpose, a few additional site response analyses were performed with NERA - a software specially designated for such dynamic applications. Two values of PGA (peak ground amplitude) were considered depicting two different scenarios: strong and medium earthquake signals. The results were expressed as 
	  
	 Thus, the current study applies various numerical methods to achieve same output and proceeds with a comparison between the analyses. It acknowledges, in the same time, the differences and limitations occurring amid. Given the fact that the study deals with a non-linear kinematic hardening process, it becomes quite difficult to point exactly the main cause that leads to dissimilarity. The overall results show a fair match when comparing the two software, but also when accounting for external sources KNMI, \cite{dost2012monitoring}, \cite{dost2004scaling}, \cite{dost2013august} that carried out particular investigations. 
		
	It can be concluded that for the Groningen sites, low PGA input ($a_g=0.1g$) the surface acceleration response amplified whereas when dealing with a high PGA input ($a_g\textgreater0.15g$) the top PGA is de-amplified. Both the strength properties of the homogeneous medium and the earthquake motion characteristics influence the seismic wave propagation though the soil; for this particular case, the inelastic response depends on the limited capability of the soft layers to transmit seismic signals to the surface because of the low strength and on the high hysteretic damping observed at large strains. At low PGA values, the soil displays elastic behaviour as it reaches the surface, the frequencies do not experience considerable damping whereas at high PGA levels, the soil shifts towards its natural frequency, damping the higher ones and non-linear constitutive model plays a more important role than the viscous damping effect.