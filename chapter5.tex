	\chapter{Soil Structure Interaction analysis} \label{ch6}
	
	The current chapter presents the analysis performed in 2D plane stress domain of the soil-footing(-structure) interaction when subjected to a recorded acceleration-time history. Various methods of seismic implementation are carried out to examine aspects such as load effects on soil response, effectiveness of energy dissipation mechanisms and the occurrence of foundation uplift.
	
	\section{New design philosophy}
	As previously presented, the new design philosophy suggests a different perspective on the induced non-linearity over the soil-structure assembly. Researchers involved within this study (Anastasopoulos et al., 2010 \cite{anastasopoulos2010soil}, Gazetas et al, 2013 \cite{gazetas2013nonlinear}, Apostolou et al., 2011 \cite{apostolou2011soil}, etc.) showed \mbox{several} effects associated with soil "failure" as acting beneficially in case of seismic ground motion. The interest in the present essay is to apply such approach on a different setting: soft soils, lack of bedrock, induced, and not tectonic, earthquakes together with shallow foundations. In order to summarize the goals, it is utmost important to define the characteristic non-linear effects:
	\begin{itemize}
		\item separation at soil-footing surface under rocking vibration also known as \textit{uplifting} that occurs when negligible interface tensile capacity is considered.
		\item mobilisation of the bearing capacity type of failure mechanism  when experiencing large cyclic overturning moments - \textit{soil failure}.
		\item plastification of the underlying ground in the proximity of the edges of the foundation \mbox{induced} by large vertical stresses.
	\end{itemize} 
	
	\section{Problem definition}
	The former chapter gets extended towards a 2D plane stress problem involving a soft soil layer supporting a shallow foundation. The entire structural system is subjected to a seismic input motion. The study aims at inspecting the inelastic soil response, the proposed method accuracy and its limitations. The final goal is to be able to acknowledge the new design philosophy suitability for this specific earthquake in Groningen area.
	
	\section{Model description}
	The object of study of the current chapter includes a soil layer of 30 meters deep together with a simplified soil-structure system resting upon it. As expected, the soil properties do not change drastically, for all the work that was previously conducted proves the parameters validity. Thus, the preceding model (1D soil column, see chapter 4, section 4.2.2) is expanded into a larger soil layer supporting the shallow foundation. A chimney located in Hoogezand village, Groningen, in  the north of Netherlands serves as example for the building and it was chosen for its relatively simple geometry together with available structural properties. The location and the chimney itself can be seen in Figures \ref{Boom} and \ref{chimney}.
	
	\begin{figure}[!h]
		\centering
		\includegraphics[width=0.5 \linewidth]{"Boomgaard"}
		\caption{Map location of the chimney - Groningen area}
		\label{Boom}
	\end{figure} 
	
	\begin{figure}[!h]
		\centering
		\includegraphics[width=0.5 \linewidth]{"chimney"}
		\caption{Real structure objected to study}
		\label{chimney}
	\end{figure} 

	
\section{Method of analysis}
The inelastic soil response investigation is implemented numerically with a finite element model considering two-dimensional plane strain assumptions. Both soil medium and \mbox{footing} are modelled as deformable solids using quadrilateral, continuum elements. The chimney is represented by a rectangular footing with a lumped mass on top of a slender, rigid pier. Moreover, the footing is also considered to behave in a stiff fashion when seismically loaded. The main emphasis is on the soil response rather than on the structural one, thus the simplification. A state-of-art contact algorithm defines the soil-structure interface incorporating uplifting and sliding features whereas purely elastic impact assumption is adopted. Every analysis starts with a \textit{static step} - the geostatic step in which initial stress conditions are established.  

The dynamic analyses operate through an implicit direct-integration algorithm - the global set of non-linear equations of motion is integrated in time domain via the implicit Hilber-Hughes-Taylor operator (a customed Newmark's method). As it was mentioned before, the Newmark's method includes the $\alpha$ and $\beta$ parameters with the values explained in the previous chapter (see section 4.2.4, chapter \ref{ch3} and Appendix \ref{App:AppendixH}). The numerical stability is achieved by calibrating both the mesh size and the time-step increment in such a way that elastic wave propagation laws are not violated; the calibration is also explained in Appendix \ref{App:AppendixI}. A schematization of the FE model can be seen in Figure \ref{mainM} followed by a detailed description of the model.

\subsection{Analysis procedure}
The research strategy resembles to the ones previously presented (see section 4.2.1). Each analysis begins with a geostatic step such that the in-situ stresses are established first. Gravity load is applied all over the soil medium, simulating the real soil status when no foundation is resting on the ground.

Further on, the footing influence is similarly introduced as the material density together with the gravity load define the foundation self weight. No additional vertical load is required so far. Moreover, the rigid pier includes a mass element at the top that simulates the superstructure weight. In the end, the whole assembly aims to reproduce a single-DOF structure.

Lastly, the seismic excitation must be integrated within the model. In order to get a smooth transition from the Site Response Analysis (SRA) towards the final goal, the study first assumes an sinusoidal cyclic input at the base of the soil layer. Finally, the real acceleration time history presented in previous chapter serves as example. The PGA is assumed to be 0.28g and it is implemented at a depth of 30 meters. Firstly, because this location is consistent with the SRA performed in chapter 4 and secondly, because it simulates the presence of bedrock.

The boundary conditions are formulated such that it accounts for seismic wave propagation and reflection at the bounds. Numerical stability, computational expenses and model dimensions play an important role for the accuracy of the output and require most attention.  %Henceforth, different types of boundary conditions were imposed that mimic the acceleration time history in distinct manners. Important to keep in mind that some methods do not necessarily require dynamic algorithms, a quasi-static analysis is preferable for such cases - this aspects will be explained in detail later in this chapter.

Thus, there are three main steps conducted within all analyses:
\begin{enumerate}
	\item \textbf{Geostatic} - for setting up the initial stress conditions of a free-field;
	\item \textbf{Self weight application} - representing the perturbation on the stresses close to the surface due to the presence of the footing;
	\item \textbf{Dynamic/ Quasi-static} - applying the seismic input in various manners to simulate the earthquake motion.
\end{enumerate}

\subsection{Contact definition}
An important role in triggering uplifting is played by the contact definition. The user can create and customize the interface properties according to the desired application. In finite element analysis, contact conditions describe a particular category of discontinuities, enabling forces to be transmitted from one part to another. The model must recognize and distinguish when the parts are in contact and when separation occurs in order to apply the constraints properly. 

Abaqus offers the possibility of defining contact pairs or contact elements; the former is recommended and it is based on the master-slave formulation. In addition, it allows the definition of these entities as surfaces or a collection of nodes - the differences refer to methods of discretization and calculation. In order to define a contact, its interaction properties have to be formulated first. Again, Abaqus allows the user to customize these features in correspondence with the model in use. For this particular case, only mechanical properties were created. An advanced contact algorithm is applied in order to permit uplifting and sliding and, additionally, to control their development during the seismic excitation. 

As previously mentioned, the study investigates two types of soil-footing contact: \textit{fully bonded} and \textit{tensionless sliding interface}. The advanced algorithm referes to the sliding interface whilst the fully bonded contact is effortlessly depicted through a \textbf{Tie constraint} - using the same master-slave formulation, it provides a simple way of bounding the two surfaces permanently, preventing sliding/separation.

However, the tensionless sliding interface requires much more attention and research as it incorporates various features that will be detailed further on.
 
\paragraph{Mechanical properties}   
The contact is described for both normal and tangential direction, more precisely a pressure-overclosure relationship illustrates the normal behaviour whereas a frictional interaction deals with the tangential behaviour. 

For normal direction, Abaqus implements a "hard-contact" relationship by default which consists of:
\begin{itemize}
	\item no penetration of master surface into the slave;
	\item no upper bound for the transmitted contact pressure;
	\item no stress transfer between surfaces when there is no contact detected.
\end{itemize}

Nonetheless, the current study indicates another type of normal contact, a softened one involving a pressure-overclosure relationship. According to Abaqus documentation, the softened contact can be introduced as linear or exponential, the latter being preferred. In an exponential (soft) contact pressure-overclosure relationship the surfaces begin to transmit contact pressure once the clearance between them, measured in the contact (normal) direction, reduces to $c_0$. The contact pressure transmitted between the surfaces then increases exponentially as the clearance continues to diminish. The schematization of the relationship can be visualized in Figure \ref{pressure}.
\begin{figure}[!h]
	\centering
	\includegraphics[width=0.45 \linewidth]{"pressure"}
	\caption{Exponential “softened” pressure-overclosure relationship in Abaqus/Standard}
	\label{pressure}
\end{figure} 

where $p_0$ is the contact pressure at zero distance, $ c_0$ is the distance from the master surface at which the pressure is decreased to 1 \% of $ p_0$. The behaviour in between is exponential. A large value of $ c_0$ leads to soft contact, a small value to hard contact. It is difficult to know a priori which values are suitable, however literature studies suggest for $p_0=10 kPA$ correlated with a clearance $c_0=10E^{-5} m$.


For the tangential behaviour, the software provides the following options:
\begin{itemize}
	\item frictionless - default;
	\item rough - no slip is allowed;
	\item penalty friction which allows for:
		\begin{itemize}
			\item Coulomb friction law $T=\mu N$;
			\item constant shear limit at surface;
			\item friction coefficient as a function of the slip rate and contact pressure;
		\end{itemize}
\end{itemize}

Considering the cohesive character of the soil, the Coulomb friction law might not be the appropriate solution, thus the second option was chosen. This shear stress limit is typically introduced in cases when the contact pressure stress may become very large causing the Coulomb theory to provide a critical shear stress at the interface that exceeds the yield stress in the material beneath the contact surface. A reasonable upper bound estimate for $\tau_{max}$ is $\sigma_y/\sqrt{3}$, where $\sigma_y$ is the Mises yield stress of the material adjacent to the surface. In some cases some incremental slip may occur even though the friction model determines that the current frictional state is “sticking.” In other words, the slope of the shear (frictional) stress versus total slip relationship may be finite while in the “sticking” state, as shown in Figure \ref{stick}.

\begin{figure}[!h]
	\centering
	\includegraphics[width=0.5\linewidth]{"stick"}
	\caption{Elastic slip versus shear traction relationship for sticking and slipping friction}
	\label{stick}
\end{figure} 

The relationship shown in this figure is analogous to elastic-plastic material behavior without hardening: K corresponds to Young's modulus, and $\tau_{crit}$ corresponds to yield stress; sticking friction is associated with the elastic regime, and slipping friction to the plastic one. A typical average value recommended in the literature for the elastic slip is 0.1mm and it seems appropriate for this case as well.

\paragraph{Contact interface discretization}
Having defined all interaction properties, the contact itself must be formulated; the most suitable interaction option Abaqus/Standard offers is \textit{Surface-to-surface}; the contact discretization can be chosen node-to-surface (N-S) or surface-to-surface (S-S), the latter being preferable. Firstly, because the conditions will be enforced based on an average region instead of strictly node-wise. The averaging regions are approximately centred on slave nodes, so each contact constraint will predominantly consider one slave node but adjacent slave nodes. Secondly, this option does not include spikes in the pressure distribution along the surface, as it happens in the N-S case. Lastly, large and unintended penetration of master nodes into slave does not occur leading to a smoothing effect.

The sliding formulation required for the interaction implementation has two options as well: finite sliding or small sliding. The second alternative is more convenient, as the footing will not slide considerably. Moreover, coupling of slave nodes with their projections on master surface is calculated at the beginning of the analysis and does not change throughout the analysis, whereas for finite sliding approach, this coupling is checked and recalculated throughout the analysis.

\subsection{Boundary conditions}
The first set of BCs refers to the soil layer and it assumes the top surface as being constraint-free, whilst rigid bottom is initially defined. The lateral boundaries are both vertically and horizontally constrained. The structure (SDOF) resembles to a cantilever, the base (footing) is connected to the soil via the contact algorithm (fixed end) while the top of the pier is allowed to translate and rotate (free end).

Nevertheless, in dynamic analyses, a fixed boundary condition will lead to wave reflection at the outer limits of the model which means energy trapped inside the model. This effect becomes detrimental, therefore solutions must mitigate reflective waves propagating throughout the model. One way is to create a large enough soil body to simulate an infinite medium. Another concept relates to transmitting boundaries and many researchers proposed methods of tackling the problem, here highlighting the work of Lysmer and Kuhlemeyer, 1969 \cite{lysmer1969finite} and Zienkiewicz et al (1989) \cite{zienkiewicz1989earthquake}.

However, for simplicity of the problem, these complex concepts were not implemented so the focus was on verifying the accurate position of the lateral boundaries. Moreover, the current application is dealing with a simple geometry, reduced  footing-structure dimensions and a dissipative material. So it can be assumed that soil plasticity will favour the energy dissipation and prevent the wave reflection at lateral boundaries to greatly influence the response in the proximity of the foundation.

 \paragraph{Tied lateral boundaries}
	The traditional approach considers pairs of lateral nodes being tied together such way that the horizontal and vertical displacement are equal throughout the entire analysis. This feature can be easily achieved using Abaqus MPC (multi-point constraint) option specifically designed for this situation. The assumption is in conformity with Zienkiewicz's work (\cite{zienkiewicz1989earthquake}) stating that the presence of the structure on soil oscillation is negligible when the boundaries are sufficiently far.
	
	Subsequently, the input motion is introduced at the vertically fixed base of the model that replicates the presence of the bedrock. As previously discussed, the acceleration signal obtained from the dataset for Groningen field was recorded at borehole depth, not at bedrock, which means that the recordings can include both incident and reflected waves. In order to exclude an overestimation of the real input signal, the accelerogram was truncated in half to ensure that only upward travelling waves are considered.
	
	Nevertheless, this method can be formulated in the FE product either through a dynamic or a quasi-static step depending whether the earthquake is expressed in terms of total displacements or total accelerations.
	
	However attractive this simple method might be, its shortcomings originate from both the lack of an actual bedrock or from the uncertainties rising when defining the model dimensions - there are no rule of thumbs for determining the truncation of the mathematical model. Thus, in order to establish a correct location for these boundaries, various tests were performed introducing the concept of free-field soil column.
	
	\paragraph{Free-field soil column}
	It behaves similarly to a vertical soil medium located far enough from the actual foundation for the vibrations not to perturb its state stress. This can be simply implemented as the one-dimensional column subjected to seismic excitation, identical to the model created for the site response analysis. Its advantages involve the option of extracting the results at any desired node. Thus, the previous analysis becomes essential as it represents the benchmark when verifying the soil medium dimensions.
		
	It is worth to acknowledge the constraints which link pairs of lateral nodes in order to maintain equal horizontal displacements along the seismic test. Then, it suffices extracting the outcome from one set of lateral nodes, since left or right are displaying equal results. The output yields displacement, velocity and acceleration time histories for each lateral node - allowing the comparison with the main model lateral response.
			
	A series of results extracted from the SRA are presented below:
		\begin{figure}[h!]
			\centering
			\includegraphics[width=0.8\linewidth]{"response_model_v12_FBC"}
			\caption{Acceleration, velocity and displacement time histories output throughout the free-field soil column}
			\label{Acc_ff}
		\end{figure} 
		
	Once the response at the lateral boundaries of the main model corresponds to the one recorded from the free-field soil column, then the location of the boundaries becomes valid and the final dimension of the model can be established. For this particular case, it was concluded that a total length of the soil medium, L=100m, is sufficient to incorporate both near and far field domains. Figure \ref{validation}) shows a comparison between the response obtained from both one-dimensional model and the main 2D model.
		\begin{figure}[!h]
			\centering
			\includegraphics[width=0.7\linewidth]{"free-field2"}
			\caption{Validation for soil layer width by comparing the acceleration response at surface level}
			\label{validation}
		\end{figure}			
			
			
	\newpage
	\paragraph{Main model}
	After setting the correct dimensions, the main model, as it can be seen in Figure \ref{mainM}, consists in few features such as:
	\begin{enumerate}
		\item \textit{soil medium} - a homogeneous clay layer, solid, deformable, displaying non-linear kinematic hardening. Same material and meshing (plane-strain CPE4R) properties as the free-field column (see also Appendix \ref{App:AppendixA} and \ref{App:AppendixF}).
		\item \textit{footing} -  rectangular solid (B=4m, b=0.5m), deformable, meshed with continuum elements (plane-strain with reduced integration CPE4R) behaving elastically.
		\item \textit{pier} - beam element (B21), perfectly elastic representing the chimney itself.
		\item \textit{mass} - concentrated mass represented by a point assigned with inertial mass. Together with the footing and pier, the assembly is in conformity with the assumption of a SDOF system.
		\item \textit{contact interface} - special algorithm contact presented in detail in section 5.4.2;
	\end{enumerate}
	
		\begin{figure}[!h]
			\centering
			\includegraphics[width=0.9\linewidth]{"mainmodel"}
			\caption{Schematization of the main FE model}
			\label{mainM}
		\end{figure}
		
	It is worth mentioning few assumptions related to wave propagation which simplify the problem without suffering from loss of accuracy:
	\begin{itemize}
		\item only vertically propagating seismic waves were taken into account; firstly, because a distinctive feature of this specific induced earthquake are the dominant shear waves. Secondly, because the free-field soil was also subjected to S-waves exclusively. And thirdly, because the two body waves are independent of each other. Thus, the compressive waves do not influence the soil-structure behaviour for now.
		\item the body waves travel towards the lateral boundaries under an incidence angle of $\theta = 0$. Basically, there are no \textit{evanescent waves} within the deformable body, waves that occur due to combinations of boundary conditions and, unlike the P and S-waves, are frequency dependent.
	\end{itemize}
	
\subsection{Cyclic analyses}
	The investigation starts with a series of  uniform harmonic cyclic tests because the trends in response are far easier to observe in such conditions. The method, as it was described previously (see section 2.7.3), represents a fair simplification of the seismic \mbox{problem}. Also, the researcher community (Gazetass et al, \cite{gazetas2004seismic}, Drosos et al, 2012 \cite{drosos2012soil}) took similar steps in order to observe the evolution of settlement-rotation response of the foundation. Moreover, the current study can be regarded as rather exclusive, since the seismic inspection performed for Groningen site conditions corresponds to the conventional design (available within EC 8). Thus, for verification, this paper compares the obtained results with the ones accessible in the scientific literature.

	%A simple rectangular footing with dimensions L=4m and b=0.5m is resting on the non-linear soil medium. %As mentioned before, the analysis investigates the differences between two types of soil-footing contact: FBC and TSI. The fully bonded contact assumes the elements are permanently tied together during the analysis whilst the sliding interface allows uplifting to occur. 
	%The advanced contact algorithm was described in previous sections and the output consists of responses extracted from both analyses. The footing acts like a rigid body; the boundaries are located at considerable distance from the footing in order to avoid its influence on the stress field.
	The test starts with a geostatic step, followed by the appliance of the foundation-pier self weight and sinusoidal cyclic excitation at the base of the soil layer. 
	
	The dead load acting on the building is converted into a concentrated force applied at mass level. One of the main goals of this study is to investigate the difference in response between lightly and heavily loaded foundations. To distinguish the two situations, the concept of vertical safety factor ($FS_v$) is introduced, which relates the applied axial force (at mass level) to the footing bearing capacity.
	
	Based on Terzaghi, 1943 initial work, Meyerkof, 1957\cite{meyerhof1957ultimate} proposes the ultimate bearing capacity of a rectangular footing only for vertical loading and in undrained conditions as:
	\begin{equation}
		s_c=1+B/L*0.2
	\end{equation}
	\begin{equation}
		q_{ult}=(\pi +2)S_u*s_c + q \longrightarrow q_{ult}=1.2 (\pi+2) 10= 61.8kPa
	\end{equation}
	\begin{equation}
		N_{ult}=q_{ult}xB^2=989 kN/m^2
	\end{equation}
	where $S_u$ is the undrained shear strength underneath the foundation (here, $S_u=10kPa$), $B$ and $L$ are the width and length of the foundation (here $B=L=4m$), $q$ is the surcharge (here, initially $q=0kPa$), $q_{ult}$ is the distributed ultimate load which leads to the $N_{ult}$ - ultimate bearing capacity. Now, considering only 2-D plane strain conditions, the value is calculated for a cross section, yielding:
	\begin{equation}
		N_{ult}=q_{ult} B \longrightarrow N_{ult}=247.25 kPa
	\end{equation}
	
	Additionally, a static numerical analysis was performed in Abaqus to verify the value of the ultimate bearing capacity for vertical loading. The result can be extracted using a load-displacement plot which can be seen in the figure below.


		\begin{figure}[!h]
			\centering
			\includegraphics[width=0.5\linewidth]{"ultimatebearing"}
			\caption{Load-displacement representation calculated numerically for a shallow strip of L=4m}
			\label{bearing}
		\end{figure}

	Thus, the gross allowable-load bearing capacity of shallow foundations requires the application of a static vertical factor of safety ($FS_v$) to the gross ultimate bearing capacity:
	
\begin{equation}
	FS_v=\frac{N_{ult}}{N_{allow}}
\end{equation}

Subsequently, the dynamic input signal is represented by an uniform sinusoidal cyclic curve, consisting of 10 cycles of 1Hz frequency and an acceleration of 1g. This is relatively close to the recorded acceleration which has an overall frequency of 1,4Hz. The signal is applied at the base of the soil medium, in the horizontal direction whilst the translation in vertical direction is constrained.

%Thus, there are two analyses performed: the lightly loaded foundation corresponding to a vertical safety factor $FS_v=5$ and the heavily loaded one assigned with $FS_v=2$. The results of the analysis together with several remarks associated with this uniform sinusoidal acceleration signal can be found in the following section.
%\color{red}Furthermore, the cyclic push-over test consists in a controlled lateral displacement of the mass point assigned to the top of the pier. It simulates the seismic action as the deformation is assigned with a harmonic cyclic amplitude curve. The same approach was presented formerly (see sections 3.3.1. and 4.4). The input curve consists in 10 cycles with a frequency of 1Hz and an acceleration of 0.28g. Again, Abaqus combines the prescribed displacement with the curve yielding a slow-cyclic movement. 
%\textit{NOTE: This analysis is currently in stand-by as I had troubles using these features in Abaqus.}
	  
\subsection{Seismic analyses}	 
Once the whole concept is grasped and a stable model is obtained, the paper \mbox{continues} with the dynamic step for which the acceleration time history is represented by the recorded data. The Groningen earthquake signal is scaled to include a PGA of 0.28g as representative for this specific site. A sketch of the input, expressed in $m/s^2$, can be seen in the figure below.
	
	\begin{figure}[!h]
		\centering
		\includegraphics[width=0.7\linewidth]{"input_acc"}
		\caption{Acceleration input for PGA=0.28g trimmed to half in order to include upwards travelling waves exclusively. The signal is divided into periods of 2 seconds each - the colours will correspond to the results presented in the next section.}
		\label{inputacc}
	\end{figure}
\pagebreak

\newpage
\section{Results}
This section focuses on the results, on several components and conditions which might influence the outcome differently. A first important aspect is the static safety factor for vertical loading $FS_v$ which makes a distinction between \textit{lightly} and \textit{heavily loaded foundations}. Furthermore, the presence of a surcharge in the proximity of the footing is also examined. Moreover, the structural geometry effects on the non-linear soil response are addressed. One other feature was examined focusing on the influence of the natural frequency of the soil layer and of the structure. Finally, the type of contact at soil-footing interface is analysed and the comparison between the obtained values and the ones found in scientific literature is conducted.

The outcome first indicates the response according to the uniform cyclic input, in order to observe the trends and compare with the literature study, followed by the response of the seismically loaded system. Each aspect will be described and remarks will be formulated.

Keep in mind that Figure \ref{inputacc} contains a series of colours which represent a period of t=2 seconds of the acceleration time- history each. The colours should be regarded as conventional for all the upcoming results. It is preferred to adopt such display due to non-linear response the soil-structure assembly but also because it is easier to follow the development of stresses and deformations.

\subsection{Effects of vertical loading}
Studies show (Gajan, 2009 \cite{gajan2009effects} Drosos et. al, 2012 \cite{drosos2012soil}), Anastasopoulos, 2014 \cite{anastasopoulos2014simplified}) that the safety factor plays an important role when it comes to distinguish the uplifting from soil yielding. Two situations are discussed for this part: lightly and heavily loaded foundations. The first assumes a low level of vertical loading acting on the shallow footing which translates into a large static vertical safety factor, $FS_v=5$. The latter represents the opposite with a safety factor $FS_v=2.$ The comparison between the obtained results and the scientific literature (explained in Section 2.7.2) makes this check possible.

Figures \ref{sin} and \ref{acc4msin} show the output corresponding to the uniform sinusoidal \mbox{acceleration} input of 10 cycles of frequency 1Hz. The moment-rotation ($M-\theta$) and settlement-rotation($w-\theta$) results are extracted from the foundation midpoint. It offers relevant information regarding the foundation behaviour as it acts as intermediary between the superstructure and the soil medium. The outcome corresponds to the last step of the analysis (dynamic), so the settlement does not include the contribution of dead load.

Apart from the inelastic response extracted at foundation midpoint, the cyclic behaviour of one element beneath the centre of the footing was studied in terms of hysteresis loops (see Figure \ref{sin} (c1) and (c2)). The current thesis started with a strain-controlled cyclic shear test on one element mesh (see section 3.5). The purpose was to observe the cyclic behaviour of the material model and to extrapolate it to larger soil medium. The responses detected in both the analyses (the current and the one element mesh) show similar trends of the hysteresis loops. This hints that the wave propagation occurs in an accurate way throughout the soil layer and that the soil is, indeed, under cyclic loading.

As for the acceleration response, the emphasis is on three points across the model: the mass level (top of the pier), surface (underneath the midpoint of the footing) and base level (bedrock). The purpose of such examination is to observe whether the structure experiences amplification or de-amplification which can hint if the ductility demand decreased or not. Additionally, the amplification factor is investigated in the same fashion previous chapter operates (see section 4.4).

	\begin{figure}
		\centering
		\includegraphics[width=1\linewidth]{"sin_4m"}
		\caption{Sinusoidal cyclic test results pointing the \mbox{differences} between lightly loaded foundation $FS_v=5$ and heavily loaded foundation ($FS_v=2$): (a1),(a2) moment-rotation; (b1), (b2) settlement rotation at foundation level; (c1), (c2) hysteresis loops right below the centre of foundation. The results correspond to a shallow footing of 4m width and two different dead loads.}
		\label{sin}
\end{figure}

\begin{figure}[!h]
	\centering
	\includegraphics[width=0.6\linewidth]{"drosos2"}
	\caption{Results according to Drosos et al, 2012 for a base excitation of a 12-cycle 1Hz sine with 0.5 acceleration amplitude. The response is expressed in terms of moment-rotation (top) and settlement-rotation (bottom). Note that small foundation($FS_v=3.5$) and large foundation ($FS_v=7.3$)}
	\label{drosos}
\end{figure}

The prevailing method of analysis resembles the work conducted by Drosos et al, 2012 \cite{drosos2012soil} who \mbox{performed} a series of tests including the harmonic input excitations. The authors \mbox{examined} the behaviour of three foundations with various dimensions making the \mbox{difference} between small, medium and large foundation. Figure \ref{drosos} presents the results according to a base excitation of 1Hz amplitude with an acceleration of 0.5g.

It is important to mention that the researchers distinguish the different vertical safety factors and aspect ratios by keeping the superstructure dead load and soil properties constant while changing the foundation size. So, the large foundation corresponds to the \mbox{conservatively} designed footing whereas the small one indicates the \textit{under-designed} foundation. This also hints the differences between the capacity design and the new \mbox{philosophy} design. Moreover, assuming constant dead load, it means that the large foundation corresponds to lightly loaded foundation while the small one represents the opposite. 

Meanwhile, the current study proposes two ways of differentiating the separate cases: firstly, it keeps the foundation size and soil properties constant while changing the dead load. \mbox{Secondly}, it follows the aforementioned approach, changing the foundation size. The former case is \mbox{presented} within this section, the latter will be discussed in the following one.

\begin{figure}[!h]
	\centering
	\includegraphics[width=11cm,height=5cm]{"acc_sin4m2"}
	\caption{Acceleration response at three main points within the model for both lightly loaded foundations ($FS_v=5$) and a comparison between the two footings at mass level. Both the foundations display similar acceleration trends, so only one was chosen as representative.}
	\label{acc4msin}
\end{figure}

\newpage
Before elaborating the remarks regarding the obtained results, it is crucial to acknowledge the dissimilarities between the current study and the literature ones chosen to serve as benchmark. Several aspects leading to different response are explained as it follows:
\begin{enumerate}
	\item most of the investigations found in the literature include a soil layer with constant \mbox{properties} along its depth while the current study deals with linearly increasing soil characteristics.
	\item the soil conditions in Groningen reveal a rather soft soil compared to the other ones. The research community suggests a stiff clay medium with undrained shear values around 75kPa while the Dutch soil presents much lower values (i.e. 5-10kPa at surface). Moreover, the stiffness properties seem to be lower as well.
	\item the height of the layer is set to 30 meters in this paper whereas most of the scientific studies incorporate layers of maximum 10 meters. However, it was preferred to continue with such "deep" layer in order to be consistent with the previous numerical step (see chapter 4 - SRA). The dynamic motion has a great influence on the non-linear soil response. Nonetheless, the location of the bedrock plays an important role since the seismic waves are dispersed as they travel upwards.
	\item the earthquake signal introduced at the base of the model varies as well. On one hand, there is the induced-earthquake motion corresponding to Groningen field: shear-wave dominated with low frequencies. On the other hand, there are the strongest seismic signals ever recorded with high frequencies and large PGAs or sinusoidal cyclic excitation input with medium and large accelerations. Nevertheless, there is a clear difference between the earthquake signals used in the numerical analyses. 
\end{enumerate}

After mentioning such dissimilarities between the scientific studies and the current paper, various remarks can be formulated with emphasis on the mechanical behaviour of the foundation. 


 
 \paragraph{Seismic analyses}
 However, the main interest in performing such uniform cyclic tests is to check whether the current model captures similar trends compared to the ones observed within the research community. The real earthquake input contained deleterious asymmetric pulses (compared to the sinusoidal cyclic one) that might induce significant foundation rotation. According to figures \ref{sin} and \ref{drosos} the outcome are relatively comparable. Hence, the investigation can continue towards a seismic analysis, involving the acceleration time-history recorded at Groningen site. The results and discussion will be presented as it follows.
 
 This key parameter in soil inelastic response dominates the interplay between soil uplifting and the bearing capacity type of failure mechanism. Firstly, the lower the FSv (heavily loaded) the more inelastic the response experienced by the foundation. This translates into a greater rate of settlement per each cycle and reduced uplifting. Additionally, the response of heavily loaded (or under-designed) foundations is dictated by the material nonlinearities (soil yielding) (see fig \ref{eq1} (a1, b1)) while the lightly loaded footings are governed by the geometric nonlinearities (uplifting, sliding) which dissipate a greater amount of energy (see fig \ref{eq1} (a2, b2).
 
 The obtained outcome confirms that the rate the soil settles decreases with the number of cycles due to soil densification (or to the increase in vertical stresses below the footing). As the footing settles, more elements must be engaged within the soil plastification area, thus the larger the need of elements, the lower the settlement rate. 
 
 As the safety factor decreases, the footing remains in full contact with the soil. The phenomenon can also be seen in figure \ref{opening} representing the separation at soil-foundation interface below one corner of the footing. Yet, as the heavily loaded foundation uplifts on one side, it encounters soil yielding under the opposite corner which associates with the accumulation of settlements together with permanent rotations in the case of asymmetric loading. This accumulation of settlements indicates the overstrength of the under-designed (heavily-loaded or small) foundations.
 
  \begin{figure}[!h]
  	\centering
  	\includegraphics[width=0.95\linewidth]{"eq_fs5-24m"}
  	\caption{Foundation response to base excitation corresponding to a PGA=0.28g expressed in terms of: (a) moment-rotation; (b) settlement-rotation of the footing midpoint; (c) hysteresis loops underneath the structure. Two different cases are presented: lightly loaded (left) and heavily loaded foundation (right)}
  	\label{eq1}
  \end{figure}
  
  \begin{figure}[!h]
  	\centering
  	\includegraphics[width=0.7 \linewidth]{"opening"}
  	\caption{Contact opening under the corner of the foundation for lightly loaded ($FS_v=5$) and heavily loaded foundations ($FS_v=2$)}
  	\label{opening}
  \end{figure}
 
 The lightly loaded foundation ($FS_v=5$) on the other hand, is prevented for such accumulation of deformation, with the cost of larger overturning moments. As a matter of fact, Kutter, 2010 \cite{kutter2010estimation} proposes a method to estimate the rocking capacity of the shallow foundation, with an emphasis on the overturning moment. The author states that "..the rocking footing will sit on the reduced area at which the axial force is counterbalanced by the bearing capacity of the critical contact area.". The concept of rocking moment capacity as a function of the critical contact area is presented in figure \ref{rocking_cap}.
 \begin{wrapfigure}{l}{8cm}
 	\centering
 	\includegraphics[width=7cm,height=6cm, keepaspectratio]{"rocking_cap"}
 	\caption{Critical contact length and rocking moment capacity according to \mbox{Kutter} et al, 2010}
 	\label{rocking_cap}
 \end{wrapfigure}
 
 Analytically, the capacity can be expressed: 
 
 \begin{equation}
 M_{cap,footing} = \frac{V.L_f}{2}(1-\frac{L_c}{L_f})
 \end{equation}
 
 where $V$ is the axial load applied on the structure, $L_f$ is the footing width, $L_c$ is the critical contact area. Generally, the ratio $L_f/L_c$ is estimated as the conventional static safety factor against vertical loading $FS_v$. So:
 
 \begin{equation}
 M_{cap,footing} = \frac{V.L_f}{2}(1-\frac{1}{FS_v})
 \end{equation} 	
 
 Thus, it can be argued that the larger the safety factor, the larger the rocking moment capacity of the footing. This also confirms the results obtained when analysing the three foundations (see figure \ref{eq2} (a1,a2,a3) and \ref{drosos}): the bigger footing experienced the greater overturning moment, while the smaller one encounters a reduced moment which also relates to their capacity. Simultaneously, a heavily loaded foundation experiences larger rotations which may lead to a "local overstrength" underneath one edge. For instance, in figure \ref{eq1} (b2), one can observe that the foundation rotates towards left, settling in the same time (step 1). Once the earthquake changes direction, the structure starts rotating towards right, settling again (step 2). However, the foundation rotates less with each cycle, while the deformation rate decreases as well (step 3). The phenomena is better explained in the figure below. The soil underneath the footing is behaving strongly inelastic accumulating plastic deformations progressively. The high residual rotations will generate large stresses within the superstructure as it struggles to maintain its stability. On the other hand, the lightly loaded foundation (fig \ref{eq1} (b1)) does not show such mechanism, the deformation being nearly elastic. 
 
  \begin{figure}[!h]
  	\centering
  	\includegraphics[width=0.7 \linewidth]{"bearing"}
  	\caption{Bearing capacity failure mechanisms of highly loaded foundations illustrated in few stages}
  	\label{bearingc}
  \end{figure}
 
 \pagebreak
 
 \newpage
 \subsection{Effect of the signal frequency}
 In addition, the point of interest shifted towards the importance of the seismic frequency content. Up until now, the investigation focused on a shaking motion consisting of 10 cycles of 1Hz; the additional work involves a similar uniform harmonic signal of 2Hz to observe the changes in response. Figure shows the comparison between the behaviour of the foundation when subjected to both $f=1hz$ and $f=2Hz$.
 
 \begin{figure}[!h]
 	\centering
 	\includegraphics[width=0.8\linewidth]{"2hz"}
 	\caption{Foundation response when subjected to 10 cycles of f=1Hz (left) and f=2Hz (right) in terms of moment-rotation, settlement-rotation and hysteresis loops}
 	\label{2hz}
 \end{figure}
 
 \begin{figure}[!h]
 	\centering
 	\includegraphics[width=0.6\linewidth]{"acc-2hz"}
 	\caption{Acceleration response comparison between f=2Hz (top) and f=1Hz (bottom)}
 	\label{acc2hz}
 \end{figure}

 
 \begin{figure}[!h]
 	\centering
 	\includegraphics[width=0.6\linewidth]{"sinusoidal2"}
 	\caption{Foundation inelastic response when subjected to one-cycle of frequency 1Hz (left) and of frequency 2Hz (right)}
 	\label{sinus}
 \end{figure}
 Figures \ref{2hz} and \ref{acc2hz} comprise the results of a model responding to a sinusoidal excitation of frequency 1Hz in opposition with a similar analysis incorporating a vibration of frequency of 2Hz. One remark relates to the moment-rotation plot which is interconnected with the settlement-rotation response. The lower frequency content triggers larger rotations at the foundation point together with overturning moment while the higher frequencies show an opposite trend. The latter case experiences a more rapid change of direction in motion, thus the foundation does not have the time to rotate excessively, nor encounter a great overturning moment. The seismic effects are cancelled-out as the footing decelerates, stops and rocks in the opposite direction. However, it gives the chance to accumulate more (permanent) settlements. The area comprised inside the hysteresis loops indicate that the soil underneath the foundation subjected to lower frequencies dissipate more energy compared to the high frequency. Again, the ground material does not have sufficient time to diffuse the seismic energy.
 
 As regarding the acceleration response, figure \ref{acc2hz} hints that for higher frequencies, the acceleration at the top of the pier largely increased compared to the other case. The fact that the signal maintains the uniform sinusoidal shape may suggest that the structure does not damp a significant amount of energy throughout the seismic motion.
 
 Thus, higher seismic frequencies may induce a considerable accumulation of settlement \mbox{underneath} the foundation together with a stronger acceleration at the top of the structure. Yet, such high intensity acceleration can lead to decreased deck drift due to the limited rotation of the foundation and limited damage to the superstructure.
 
 
 Based on the aforementioned results, it can be stated that there a relatively satisfying (qualitatively) match between the results found in the scientific literature and the ones obtained in this analysis. A lower safety factor yields larger overturning moments at the footing midpoint, aspect which can be noticed on both the results. Moreover, the uplifting behaviour seems dominant for the under-designed foundation with the high vertical safety factor (both $FS_v=2$ and the small foundation) while a more sinking-dominated response corresponds to the opposite situation. As expected, the rate of settlements increase with decreasing safety factor, here the heavily loaded foundation experiencing almost twice the settlement compared to the lightly loaded one.
 
\newpage
\subsection{Effects of aspect ratio}
One other important feature relates to the aspect ratio, because the response considerably depends on its geometry and, generally, bigger foundations experience larger bending moments; however, such resistance is counterbalanced by the substantial inertial forces induced by the earthquake.  This other important feature represents the ratio of the height of the structure (here h=5m) and the width of the foundation (L=B=4m). Three different shallow foundations were subjected to the real acceleration time history. The width of the footing ranged from L=2m, L=4m to L=8m while the dead load and soil properties were kept constant. It is important to make a connection between all the terms used in this analysis. Thus, the small foundation can be regarded as an under-designed one or heavily loaded whilst the largest footing corresponds to a rather conventionally design, representing the lightly loaded foundation. The results can be seen in the following figures with emphasis on the moment-rotation ($M-\theta$) and settlement-rotation ($w-\theta$), but also stresses beneath the structure or acceleration response at deck (top).


 \begin{figure}[!h]
 	\centering
 	\includegraphics[width=1\linewidth]{"fs2,4,8"}
 	\caption{Foundation response to base excitation corresponding to a PGA=0.28g expressed in terms of: (a) moment-rotation; (b) settlement-rotation of the footing midpoint; (c) hysteresis loops underneath the structure. Three different cases are presented: small (FSv=2), medium (FSv=4) and large (FSv=8)}
 	\label{eq2}
 \end{figure}

 \begin{figure}[!h]
 	\centering
 	\includegraphics[width=0.9\linewidth]{"irregular"}
 	\caption{An example of inelastic soil response showing irregular trends. The study belongs to Zafeirakos, 2011 \cite{zafeirakos2011role} and it regards a caisson embedded into the soft soil. The structural system is defined as a single dof, the study makes the difference between lightly(left) and heavily loaded foundations (right)}
 	\label{eq5}
 \end{figure}

 \begin{figure}[!h]
 	\centering
 	\includegraphics[width=0.9\linewidth]{"acc_2,4,8"}
 	\caption{Acceleration time history recorded at deck level (top pier) when subjected to a real seismic base signal for three different foundation size}
 	\label{eq3}
 \end{figure}

One aspect is worth mentioning in this section which is the maximum critical acceleration $\alpha_c$. It refers to the maximum value which can be transmitted to the mass and it is determined using Newmark's sliding block theory (the concept was described in section 2.2).

\begin{equation}
	\alpha_c=\frac{M_{ult}}{mgh}
	\label{equ}
\end{equation}

where $M_{ult}$ - ultimate moment capacity of the foundation ($kNm$), $m$ - mass ($kg$), $h$ - height of the pier ($m$) and $g$ - gravitational acceleration ($m/s^2$).

It was argued previously that there is a correlation between the vertical safety factor $FS_v$ and the ultimate capacity of the footing $M_{cap, footing}$ or $M_{ult}$. Additionally, the aspect ratio also relates to the safety factor. Considering the above equation (\ref{equ}), it becomes evident the link between the maximum critical acceleration and the slenderness ratio. Thus, the larger the ultimate capacity of the footing (the larger the foundation itself), the more intense shaking the superstructure experiences. Clearly, the smaller foundation (more heavily loaded) sets a limit to the transmitted acceleration to the mass level. Theoretically, the larger (stiffer) footing involves greater acceleration amplitudes at the top as it presumably vibrates with a frequency closer to the input excitation. On the other side, the smaller foundation shows a dynamic attenuation mainly due to the strongly inelastic response.
Figure \ref{eq3} confirms the remarks showing a lower level of acceleration at deck level corresponding to $FS_v=2$ compared to the larger foundations. However, the conservatively designed foundation ($FS_v=8$) displays similar levels of acceleration as the under-designed one. This might indicate that both the footings perform in an adequate manner when subjected to a moderate seismic excitation. 

%In addition, the acceleration was converted into frequency domain to further obtain the amplification factor for each situation and to draw a series of conclusion based on this aspect. Acceleration outcome is transferred in frequency domain via a Fast Fourier \mbox{computation} for responses from soil surface and lumped mass; the transfer function is calculated:

%\begin{equation}
%A(f)=\frac{F_{a,mass}(f)}{F_{a,surface}(f)}
%\end{equation}
%where $A(f)$ - site amplification factor; $F_{a,mass}(f)$ and $F_{a,surface}(f)$ - Fourier amplitudes of the lumped mass on top of the pier and of the ground acceleration at surface, respectively. 

%\begin{figure}[!h]
%	\centering
%	\includegraphics[width=0.9\linewidth]{"fourier_eq"}
%	\caption{Amplification factor in frequency domain at deck level (top pier) when the structure is subjected to a real seismic base signal for three different foundation size}
%	\label{equuu}
%\end{figure}
After all, one may deduce that the rocking isolation mechanism encountered by the smallest foundation approves the new design philosophy feature: taking advantage of the non-linear soil response will lead to a reduction in the superstructure ductility demand. 


\subsection{Effects of surcharge}
In addition, the influence of structures in the proximity of the soil-foundation system was accounted for. A surcharge ranging from $q=10kN/m$ to $q=30kN/m$ was added at a distance of 1 meter from the footing with a total span of L=5m. It represents a lighter building, as the one seen in Figure \ref{comp}. No major effect on the inelastic response was observed other than increased vertical stress associated with settlements underneath the surcharge area (which was expected either way). Figure \ref{maxax} show the little variance in response of both foundation (moment-rotation plot on the right) and soil (settlement-rotation plot on the left). The last two seconds were selected to represent the moment-rotation response because the difference it was easier to observe on a smaller portion of time.

	\begin{figure}[!h]
		\centering
		\includegraphics[width=0.9 \linewidth]{"surcharge2"}
		\caption{Foundation response for various surcharge values expressed in terms of moment-rotation (the last 2 seconds) and settlement rotation (t=10 sec) of the footing midpoint}
		\label{surch}
	\end{figure}

	\begin{figure}[!h]
		\centering
		\includegraphics[width=0.9 \linewidth]{"surcharge"}
		\caption{Surcharge effect in terms of vertical stresses $\sigma_v$ (top) and settlements $w$ underneath the footing}
		\label{maxax}
	\end{figure}
	
\newpage
\subsection{Fully Bonded Contact vs Tensionless Sliding Interface}
All the analyses included the tensionless sliding interface (TSI), defined through the special contact algorithm which allows uplifting and sliding (see section 5.4.2). And so far, this approach proved successfully, especially for the under-designed (small) foundations. The examination now focuses on the situation in which the footing is permanently attached to the soil - a so-called \textit{Fully Bonded Contact} (FBC). The two analyses were performed on the same foundation geometry, with a width of L=2m and for the same static safety factor ($FS_v=2$) which depicts the heavily loaded foundation (or under-designed). Figure \ref{fbctsi} presents the results in the same manner as before:

 
 \begin{figure}[!h]
 	\centering
 	\includegraphics[width=0.8 \linewidth]{"fbctsi2"}
 	\caption{Foundation inelastic response for a base excitation of PGA=0.28g considering Tensionless sliding interface (right) and Fully bonded contact (left) in terms of moment-rotation(top) and settlement-rotation (bottom)}
 	\label{fbctsi}
 \end{figure}

One remark can be made regarding the area enclosed by the moment-rotation curve because physically, it represents the dissipated energy during cycles reversals. Thus, when preventing detachment of the foundation from the supporting soil, the structure experiences larger levels of energy to be transmitted towards the oscillator. Meanwhile, the tensionless sliding interface, limits the overturning moment at foundation midpoint acting rather beneficial.

The ductility demand is strongly correlated to the overturning moment and the curvature at the base of the SDOF. Furthermore, the limited capacity of the footing sets a well-defined limit on the magnitude of the inertial forces that can develop within the structure. Thus, it can be stated that allowing geometrical nonlinearities (sliding, uplifting) to develop may improve the performance of the foundation compared to a permanently tied footing to the soil. Of course, the price to pay for such diminished energy transmitted towards the superstructure is the relatively large permanent settlement. 

\subsection{Effects of bedrock position}
It was stated before that the acceleration response at surface levels was greatly influenced by the bedrock position. The base of the model was previously represented by a series of soil nodes located at a depth of 30 meters assigned with the acceleration time-history (see fig. \ref{inputacc}). Additionally, it was speculated that the highly non-linear response at midpoint foundation is caused by the irregular pattern in energy dissipation along the soil medium. Hence, a smaller soil layer was addressed having the bedrock located at 10 meters deep. The same acceleration was injected at the bottom of the model while the soil properties (including length of the layer) were consistent with all previous work.

Figures \ref{10m} and \ref{10m2} show the inelastic soil-foundation-structure response in the same terms presented so far. Two different models were examined depicting a vertical safety factor of ($FS_v=2$) associated with the small foundation and one with ($FS_v=4$) for the medium foundation, respectively. In addition, a comparison was performed with the results corresponding to the same set of parameters, but for the deeper soil layer (H=30m). The dead load and analysis steps were kept constant for all analyses.

    \begin{figure}[!h]
    	\centering
    	\includegraphics[width=1\linewidth]{"10m_F2"}
    	\caption{Comparison of foundation inelastic response for a base excitation of PGA=0.28g considering a layer height of H=10 m(left) and H=30m (right) in terms of $M-\theta$ (a), $w-\theta$ (b) and $\tau-\gamma$. Both the cases correspond to a $FS_v=2)$}
    	\label{10m}
    \end{figure}
    
  
  \begin{figure}[!h]
  	\centering
  	\includegraphics[width=1 \linewidth]{"10m"}
  	\caption{Comparison of foundation inelastic response for a base excitation of PGA=0.28g considering a layer height of H=10 m(left) and H=30m (right) in terms of $M-\theta$ (a), $w-\theta$ (b) and $\tau-\gamma$. Both the cases correspond to a $FS_v=4)$}
  	\label{10m2}
  \end{figure}

A sudden change in the moment-rotation curve shape is observed for the smaller layer. The trend in settlement-rotation response pertains while the wave propagation is, again, simulated in the appropriate manner. Both the cases show a smoother curve and less variance in the non-linear response within the foundation. Additionally, they show also a fair match with the scientific literature findings.

One example collected from the literature refers to the work of Anastasopoulos et al, 2011 \cite{anastasopoulos2011simplified} corresponding to both numerical analyses and centrifuge tests on a clay layer. The soil is stronger compared to the current situation ($S_u=100kPa$, $PI=50$ and $p_{ult}=546kPa$)  and the numerical test is a displacement-controlled quasi-static analysis. 

\begin{figure}[!h]
	\centering
	\includegraphics[width=0.8 \linewidth]{"example4"}
	\caption{Foundation inelastic response $M-\theta$ (left), $w-\theta$ (right). The problem was analysed with a 3D model subjected to a lateral cyclic push-over test.  The centrifuge experiment performed the same test. The ultimate bearing capacity lead to a  $FS_v=2.8)$}
	\label{gazii}
\end{figure}

Another example is presented in Figure \ref{gaza} that displays the results obtained by Gajan \& Kutter, 2009 \cite{gajan2009effects}. A similar experimental lateral cyclic test was performed on a shear-wall footing structure with comparable parameters. The results were extracted at base centre point of footing, the vertical safety factor $FS_v=5.2$ and a foundation width L=2.8m.

These numerical and experimental tests are another (simplified) approach of the same problem -a dynamic analysis with emphasis on the foundation inelastic response. It is considered thus valid to use it for the verification of the obtained results.
 
 \begin{figure}[!h]
 	\centering
 	\includegraphics[width=0.5 \linewidth]{"gazetas"}
 	\caption{Foundation inelastic response $M-\theta$ (left), $w-\theta$ (right). It corresponds to experiments involving a shear-wall shallow footing subjected to slow lateral cyclic loading conditions. The cyclic motion is induced by an acuator placed at the bottom of the sample.}
 	\label{gaza}
 \end{figure}

Such change influences the natural frequency of the soil medium, as it is a function of both shear wave velocity and layer height. According to eq. 4.7 (section 4.2.3):
\begin{equation}
	f_s=\frac{4H}{V_s}
\end{equation}

with a new average shear wave velocity of $V_s=81.6 m/s$ and the layer $H=10m$, the natural soil frequency and natural period yield:

\begin{equation}
	f_s=\frac{4*10}{81.6} \longrightarrow f_s=0.5 Hz
\end{equation}
 
 \begin{equation}
 	T_n=\frac{1}{f_s} \longrightarrow T_n=2 sec
 \end{equation}

It can be stated that the soil will resonate with a seismic signal with lower frequency than the one used in calculations so far.

One remark relates to the energy dissipated along the soil medium; the lesser the travelled distance by the wave, the more energy will be transmitted to the footing. This phenomena is also represented by the moment-rotation curve which not only shows higher overturning moment values but also encapsulates a larger area. As it was concluded before, a greater area translates into more energy transmitted towards the superstructure. Figure \ref{10} show the acceleration response recorded at the top of the pier. It confirms that, in case the natural period of the superstructure matches the one of the soil medium, the building becomes more vulnerable to earthquakes. For this current case, the natural period of the chimney can be calculated as a function of the geometric characteristics of the pier cross section and the elastic properties of the reinfoced concrete:

\begin{equation}
	f_s=\frac{1}{2 \pi} \sqrt{\frac{k}{m}}
\end{equation}

where k is the lateral stiffness of a flexible beam with mass and m is the structural mass.

\begin{equation}
	k=3\frac{EI}{L^3}
\end{equation}

with E = Young's modulus of reinforced concrete ($24x10^6 MPa$), I = moment of inertia (of a cylindric body for this case) and L = length of the pier ($L=5m$). Furthermore, the moment of inertia of the chimney along the axis of symmetry is a function of the structural mass and the radius R ($R=1.5m$):
 
\begin{equation}
	I_z=\frac{1}{2}m R^2
\end{equation}

Finally, the fundamental period of the superstructure becomes:

\begin{equation}
	T=\frac{1}{f_s} \longrightarrow T=2\pi \sqrt{\frac{2}{3} \frac{m L^3}{E mR^2}}
\end{equation}

 \begin{figure}[h]
 	\centering
 	\includegraphics[width=0.7 \linewidth]{"acc_10m"}
 	\caption{Acceleration response comparison at mass level. The two situation address a (medium-) heavily (top) and a lightly loaded foundation (bottom) for both soil layer heights}
 	\label{10}
 \end{figure}

Overall, the results show a better agreement when compared to the research community findings because of the little variance between the model parameters. However, the observed trends pertain concerning the inelastic soil response influence on the soil-structure interaction. Despite the differences in values, all the conclusions formulated for other parameters (safety factor, aspect ratio and so on) remain valid.

\newpage
\section{Conclusions}
The new design philosophy aspects were examined via a 2D plane-strain model consisting in a soil layer of 30 meters depth supporting a single degree of freedom type of structure. The ground was assumed to correspond to a soft clay displaying linearly increasing properties along its depth. As for the superstructure, a chimney located in the north of Netherlands was chosen as example and it was represented by a rectangular footing together with a pier with lumped mass on top of it. The entire analysis incorporated three main steps: geostatic, self-weight of the superstructure and earthquake excitation applied at the bedrock level. 

The research started with an uniform sinusoidal cyclic base input in order to observe the trends in response. The emphasis was on the inelastic soil response and its influence on the overall soil-structure interaction. Due to the rather exclusive character of the research and lack of reliable data, the verification of the outcome was compared to the work of Gazetas et al (\cite{gazetas2004nonlinear}, \cite{anastasopoulos2010soil}, \cite{anastasopoulos2014simplified}, \cite{drosos2012soil}, etc.). The research group were among the first ones to propose the new design philosophy and to investigate its benefits and limitations. Thus, only a qualitatively comparison could be formulated to sustain the new approach suitability.

The sinusoidal cyclic input frequency ranged between f=1Hz and f=2Hz to simulate the difference amongst moderate and strong seismic signal. Furthermore, the results were associated with similar outcome from the literature study showing an overall match in terms of inelastic soil response.

Once the model was considered sufficiently reliable, the sinusoidal input was replaced with the real acceleration time history. The study focused on the moment-rotation and settlement-rotation behaviour of the foundation midpoint, the hysteretic response of the soil underneath the footing as well as the acceleration transmitted from soil surface to the top of the pier. Several important parameters describing the soil-structure assembly were systematically changed such as: the vertical static safety factor $FS_v$, the aspect ratio $h/L$, the frequency content of the base excitation, surcharge near the building $q$, the contact algorithm and, finally, the fundamental frequency of the soil layer. Summarizing all the investigated features, few remarks can be formulated:

\begin{itemize} 
	\item one of the key parameters in such investigation is represented by the vertical safety factor. It plays an important role in SSI as it governs the inelastic soil response and it distinguishes between geometric and material nonlinearities. A lightly loaded foundation experiences the phenomena known as \textit{uplifting} which is basically the detachment of the footing edge on one side and soil yielding on the other. Moreover, it records larger levels of overturning moments together with sructural rotation. On the other hand, the heavily loaded foundations display opposite trends, with a larger accumulation rate of settlements. Although it manages to dissipate more energy compared to the higher safety factor ($FS_v=5$) and to limit the inertial forces transmitted to the superstructure, the permanent settlement should definitely not be ignored. Thus, an increase in $FS_v$ leads to increased settlement beneath the footing.
	\item the safety factor is strongly correlated to the ultimate moment capacity of the footing. The larger the $FS_v$, the higher the moment capacity of the foundation and thus, greater inertial loading the structure encounters. So, the under-designed foundation (or heavily loaded $FS_v=2$) will show a reduced overturning capacity. This may be regarded as an advantage of the new design philosophy since the foundation acts as a seismic isolator for the superstructure.
	\item the mobilization of the bearing capacity type of failure mechanism corresponds to the highly loaded foundations as well as the small footings (higher aspect ratio). In this case, the impedance mechanism is rather hysteretic due to the soil inelastic response. Meanwhile, the lightly loaded foundations (or larger footings) experienced geometric nonlinearities in terms of uplifting and sliding at the soil-footing interface. The structure will rotate more, the contact area reduces during the seismic motion leading to an increased moment capacity. Hence, the ductility demand, which is a function of the overturning moment, will increase as well together with the overall cost of the superstructure.
	\item a stronger earthquake, with a higher frequency, limits the rotation the structure experiences, but it increases the accumulation rate of settlement. Moreover, the building suffers from a more intense shacking, especially at top level. 
	\item 
\end{itemize}

\section{Limitations and recommendations}


\newpage


\section{Conclusions}

