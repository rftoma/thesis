\paragraph{Theory of wave propagation}
For the simplicity of the problem, vertically propagating waves were considered exclusively; it is also well known that both the body waves - P waves and S waves - are independent of each other. According to Love(1944) \cite{love1944h}, the wave equation for an isotropic elastic medium is:

\begin{equation}
c_p^2\bigtriangledown(\bigtriangledown\cdot u) - c_s^2\bigtriangledown \times \bigtriangledown \times u = \frac{\partial^2 u}{\partial t^2}
\end{equation} 

where, according to D'Alembert equation solution, $c_p$ is the P-wave velocity ($c_p=\sqrt{\frac{\lambda + 2 \mu}{\rho}}$) and $c_s$ is the S-wave velocity ($c_s=\sqrt{\frac{\mu}{\rho}}$) with $\lambda$ and $\mu$ as Lamé constants and $\rho$ the density of the medium.

It becomes clear that when working with seismically propagating waves, the real challenge lies in the a correct implementation of artificial boundaries construction rather than defining the propagating medium. Important to mention are the two conditions required by a surface of a wave front as Nielsen, 2006 \cite{nielsen2006absorbing} states in his research:
\begin{enumerate}
	\item \textbf{kinematical} which can be written under the form:
	\begin{equation}
	\frac{1}{\cos(\theta)}\frac{\partial u}{\partial x}= \frac{1}{sin(\theta)}\frac{\partial u}{\partial y} = - \frac{1}{c}\frac{\partial u}{\partial t}
	\end{equation}
	\begin{equation}
	\frac{1}{\cos(\theta)}\frac{\partial v}{\partial x}= \frac{1}{sin(\theta)}\frac{\partial v}{\partial y} = - \frac{1}{c}\frac{\partial v}{\partial t}
	\end{equation}
	where $\theta$ is the angle of incidence between the surface wave front and the incident waves and c is the wave velocity that depending on the direction can be either associated with compressive or shear waves.
	\item \textbf{dynamical} are expressed in terms of traction forces coplanar with the wave front
	\begin{equation}
	\rho c \frac{\partial u}{\partial t}= - f_x
	\end{equation}
	\begin{equation}
	\rho c \frac{\partial v}{\partial t}= - f_y
	\end{equation}
\end{enumerate} 

In the following section \ref{implemenent} , several boundary condition concepts are presented highlighting the necessity of each. The section includes information describing the methods, differences between the studies, limitations and mathematical background. This represents one principal goal the current paper strives to achieve - an extended investigation to understand what various boundaries influence the performance of the FE model and thus decide which one is more suitable for this application.

\paragraph{Transmitting boundaries}
	Given the flaws of the aforementioned concept, it becomes necessary to incorporate features as radiation boundaries that guarantee the outgoing waves are not reflected at the boundary surfaces. 
	
	Transmitting boundaries or absorbing boundaries (\textit{ABC}) were first proposed by Lysmer and Kuhlemeyer (1969) \cite{lysmer1969finite}, followed by Zienkiewicz (1989) \cite{zienkiewicz1989earthquake}. As the name also hints, these types of boundaries are used in the form of viscous boundary tractions or dashpots in order to absorb the normal incident waves. They can be declared both globally or locally, the later proving more appealing from the numerical implementation point of view.
	
	The reason why these types of boundaries are adopted for the FE model expresses the necessity of preventing the occurence of reflected waves whilst capturing correctly the incoming waves at the boundaries. It can be regarded as a simulation of the "horizontal infinity" of a real soil layer. It proves well-suited when associated with an internal source of excitation, however it requires improvements when the input signal originates from external sources. The solution upgrade relates to the theory proposed by Zienkiewicz, also known as \textit{free-field boundary conditions} which introduces the combination between viscous boundaries and a free-field soil column.
	
	Having in mind that for a simple one dimensional, homogeneous, elastic, isotropic wave propagation problem, the free surface displacement wave equals the double of the incoming waves, it can be deducted that the input signal can be extracted from the recorded total signal on the undeformed surface. The free-field soil column allows the user to obtain sets of incoming waves parameters (displacement, velocity, acceleration) in time domain at a known position and correlate them with the viscous boundaries located at the sides of the model. As a consequence, outside the boundaries, the elastic and isotropic conditions pertain.
	 
	The theoretical background of the free-field boundary element was proposed in an efficient manner by Lysmer and Kuhlemeyer that associates the vertical free-field soil column with viscous boundary tractions. This way, the incoming wave takes the form of equivalent forces, rather than accelerations. The traction consists of two terms as it follows:
	\begin{enumerate}
		\item viscous dashpots absorb incoming waves;
		\item free-field motion simulating the undisturbed soil layer;
	\end{enumerate}
Henceforth, the equations describing the tractions, both normal $f_n$ and shear $f_s$ are:
\begin{equation}
	f_n=\rho c_p (\frac{\partial u_`}{\partial t}-\frac{\partial u}{\partial t})+l_x \sigma^`_x
\end{equation}
\begin{equation}
f_s=\rho c_s (\frac{\partial v_`}{\partial t}-\frac{\partial v}{\partial t})+l_x \tau^`_{xy}
\end{equation}
where prime quantities relate to the free-field output values, lx =-1 or lx=1 for outward normal points either negative or positive direction. First term of the equations represents the Lysmer and Kuhlemeyer traction due to the presence of dashpots whereas the second term represents the stress resulting from free-field wave propagation, including eventual static reaction forces.

Gathering all information, the modelling scheme should consist of two, perhaps three calculations as it can be seen in Figure \ref{ff} :
\begin{enumerate}
	\item one model for each column generating the free-field soil response;
	\item one main model representing the soil - footing interaction including viscous dampers at the lateral boundaries;
\end{enumerate} 

	\begin{figure}[!h]
		\centering
		\includegraphics[width=0.7\linewidth]{"free-field"}
		\caption{Real structure objected to study}
		\label{ff}
	\end{figure} 
	
%\section{Implementation of ABC} \label{implemenent}
%The aforementioned features of the free-field boundary conditions are detailed below, starting with the free-field soil column followed by the main model description together with various modelling techniques. 


, the boundary conditions change accordingly to each method of implementation.The absorbing boundaries proposed by Lysmer and Kuhlemeyer - which can be efficiently implemented as viscous dampers or dashpots assigned to lateral boundaries of the soil medium - are evaluated performance-wise in correlation with seismic loading procedures. These dampers indicate a local boundary that absorb the incoming wave - their accuracy increase proportionally with distance from the area of interest(footing) and if they are defined as frequency independent. Abaqus allows the user to specify dashpots connected to the ground - an option requiring the definition of the local DOF direction and the dashpot coefficient. In dynamic analysis the velocities are obtained as part of the integration operator; in quasi-static analysis in Abaqus/Standard the velocities are obtained by dividing the displacement increments by the time step. Additionally, it can model relative velocity-dependent force and provide a energy dissipation mechanism. %The dashpot coefficient is calculated:

%	\begin{equation}
C_1=\rho V_p A
%	\end{equation}
%	\begin{equation}
C_2=\rho V_s A
%	\end{equation}
%	where $V_s$ and $V_s$ are the compressive and shear wave velocities of the soil, $\rho$ is the soil density, A is the tributary area of the element assigned with dashpot and $C_1$, $C_2$ are the dashpots coefficients for normal and tangential direction